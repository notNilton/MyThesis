%================================================================%
%=============  Modelo de Trabalho de Conclusão de curso ========%
\documentclass[12pt, % tamanho da fonte
	%openright,	% capítulos começam em pág ímpar
	oneside, %para impressão apenas em um lado (formato digital).  		  
	% twoside, %para impressão em frente e verso.  
	a4paper,			% tamanho do papel. 
	english,			% Idioma adicional para hifenização
	brazil				% Idioma principal 
	]{packages/abntex2-pack}
	
%------------------------------------------------------------
%------------    Estrutura do texto   -----------------------         

% Pacotes Básicos:
%\usepackage{lmodern}			    % Usa a fonte Latin Modern			

\usepackage[T1]{fontenc}		  % Selecao de codigos de fonte.
\usepackage[utf8]{inputenc}		% Codificacao do documento (conversão automática dos acentos)
\usepackage{mathptmx,helvet,courier}
%\usepackage{pslatex}                % Usa a fonte Times New Roman
\usepackage{lastpage}			    % Usado pela Ficha catalográfica
\usepackage{indentfirst}		  % Indenta o primeiro parágrafo de cada seção.
\usepackage[table]{xcolor}
\usepackage{color}				    % Controle das cores
\usepackage{graphicx}			    % Inclusão de gráficos
\usepackage{microtype} 		  	% para melhorias de justificação

\usepackage{tablefootnote} % para colocar footnotes em tabelas e figuras

% Pacotes Extras:
\usepackage[final]{pdfpages}
\usepackage{amsmath,amsthm}   %Símbolos Matemáticos
\usepackage{indentfirst} % Indenta primeiro parágrafo 
\usepackage[portuguese, ruled, linesnumbered,commentsnumbered, algo2e, vlined, lined, boxed, algochapter]{algorithm2e} % Algoritmos 
\usepackage{hyperref}
\usepackage[brazilian,hyperpageref]{backref}	 % Paginas com as citações na bibliograficas


%\usepackage[num]{abntex2cite}	% Citações padrão ABNT númerico
\usepackage[alf,abnt-emphasize=bf,abnt-full-initials=no,abnt-etal-list=5,abnt-etal-text=emph,abnt-repeated-author-omit=yes]{abntex2cite}
%\usepackage[alf,abnt-emphasize=bf,abnt-etal-list=0,abnt-etal-text=emph]{abntex2cite}

% se desejar justificar as referências (pela abnt é alinhado a esquerda)
% \usepackage[alf,abnt-emphasize=bf,bibjustif]{abntex2cite}

\usepackage{float} % para ajustar a posição das imagens

\usepackage{tikzsymbols} %para caracteres emoji
\usepackage{stackengine}
\usepackage{scalerel}

\newcommand\dangersign[1][2ex]{%
  \renewcommand\stacktype{L}%
  \scaleto{\stackon[1.3pt]{\color{red}$\triangle$}{\tiny\bfseries !}}{#1}%
}

\usepackage{etoolbox}
%\usepackage[num]{abntex2cite}  % Citações numéricas

\usepackage{lipsum}

%para usar os algoritmos e matematica
\newcommand{\var}{\texttt}

\usepackage{amssymb}
\usepackage{amsmath}



% Defininfo Cores:
\definecolor{blue}{RGB}{25,25,112}
\definecolor{midgray}{gray}{.7}

\makeatletter % informações do PDF
\hypersetup{ % pagebackref=true,
	pdftitle={\@title}, 
	pdfauthor={\@author},
    pdfsubject={\imprimirpreambulo},
	pdfcreator={LaTeX with abnTeX2},
	pdfkeywords={abnt}{latex}{abntex}{abntex2}{trabalho acadêmico}, 
	colorlinks=true,     % false: boxed links; true: colored links
    linkcolor=black,          	% color of internal links
    citecolor=black,        		% color of links to bibliography
    filecolor=magenta,      	% color of file links
    urlcolor=black,
	bookmarksdepth=4 }
\makeatother
 

% -------------------------------------------- 
% Espaçamentos entre linhas e parágrafos 
\setlength{\parindent}{1.3cm} % O tamanho do parágrafo

% Controle do espaçamento entre um parágrafo e outro:
\setlength{\parskip}{0.2cm}  % tente também \onelineskip

% Definição de ambientes matemáticos em português 
\newtheorem{teorema}{Teorema}[chapter]
\newtheorem{axioma}{Axioma}[chapter]
\newtheorem{corolario}{Corolário}[chapter]
\newtheorem{lema}{Lema}[chapter]
\newtheorem{proposicao}{Proposição}[chapter]
\newtheorem{definicao}{Definição}[chapter]
\newtheorem{exemplo}{Exemplo}[chapter]
\newtheorem{observacao}{Observação}[chapter]

% Novos Comandos
\usepackage{tgtermes}
\renewcommand{\ABNTEXchapterfont}{\rmfamily\bfseries}

% Variáveis adicionais
\providecommand{\imprimirautorcite}{}
\newcommand{\autorcite}[1]{\renewcommand{\imprimirautorcite}{#1}} 
\providecommand{\imprimirsigla}{}
\newcommand{\sigla}[1]{\renewcommand{\imprimirsigla}{#1}}
\providecommand{\imprimiruf}{}
\newcommand{\uf}[1]{\renewcommand{\imprimiruf}{#1}}
\providecommand{\imprimircurso}{}
\newcommand{\curso}[1]{\renewcommand{\imprimircurso}{#1}}
\providecommand{\imprimirinstituto}{}
\newcommand{\instituto}[1]{\renewcommand{\imprimirinstituto}{#1}}
\providecommand{\imprimirdepartamento}{}
\newcommand{\departamento}[1]{\renewcommand{\imprimirdepartamento}{#1}}
\providecommand{\imprimirano}{}
\newcommand{\ano}[1]{\renewcommand{\imprimirano}{#1}}
\providecommand{\imprimirdia}{}
\newcommand{\dia}[1]{\renewcommand{\imprimirdia}{#1}}
\providecommand{\imprimirmes}{}
\newcommand{\mes}[1]{\renewcommand{\imprimirmes}{#1}}
\providecommand{\imprimirgrau}{}
\newcommand{\grau}[1]{\renewcommand{\imprimirgrau}{#1}}
\providecommand{\imprimirexaminadorum}{}
\newcommand{\examinadorum}[1]{
    \renewcommand{\imprimirexaminadorum}{#1}}
\providecommand{\imprimirexaminadordois}{}
\newcommand{\examinadordois}[1]{
    \renewcommand{\imprimirexaminadordois}{#1}}
\providecommand{\imprimirexaminadortres}{}
\newcommand{\examinadortres}[1]{
    \renewcommand{\imprimirexaminadortres}{#1}}
\providecommand{\imprimirexaminadorquatro}{}
\newcommand{\examinadorquatro}[1]{
    \renewcommand{\imprimirexaminadorquatro}{#1}}
\providecommand{\imprimirttorientador}{}
\newcommand{\ttorientador}[1]{
    \renewcommand{\imprimirttorientador}{#1}} 
\providecommand{\imprimirttcoorientador}{}
\newcommand{\ttcoorientador}[1]{
    \renewcommand{\imprimirttcoorientador}{#1}}
\providecommand{\imprimirttexaminadorum}{}
\newcommand{\ttexaminadorum}[1]{
    \renewcommand{\imprimirttexaminadorum}{#1}}
\providecommand{\imprimirttexaminadordois}{}
\newcommand{\ttexaminadordois}[1]{\renewcommand{
        \imprimirttexaminadordois}{#1}}
\providecommand{\imprimirttexaminadortres}{}
\newcommand{\ttexaminadortres}[1]{
    \renewcommand{\imprimirttexaminadortres}{#1}}
\providecommand{\imprimirttexaminadorquatro}{}
\newcommand{\ttexaminadorquatro}[1]{
    \renewcommand{\imprimirttexaminadorquatro}{#1}}

% Cria o comando \subtitulo. A norma define que o TITULO deve ser em caixa alta
% negrito, mas o subtitulo deve ser em caixa baixa.
\providecommand{\imprimirsubtitulo}{}
\newcommand{\subtitulo}[1]{\renewcommand{\imprimirsubtitulo}{#1}}

%----------------------------------------------------
\renewcommand{\imprimircapa}{  % Capa 
\begin{capa}
%\begin{center}\includegraphics[scale=1]{Figuras/ifmg.png}\end{center}
\begin{center}{
             \large \MakeTextUppercase{\imprimirinstituicao} - \MakeTextUppercase{\imprimirinstituto} \\
              %\imprimirdepartamento \\
              \MakeTextUppercase{\imprimircurso} \\
              \vspace{2cm}
			  \large {\imprimirautor} 
			  }\end{center}
\vfill
        \begin{center}
        \MakeTextUppercase{\large \textbf{\imprimirtitulo}}  \\
        \MakeTextLowercase{\textbf{\imprimirsubtitulo}}
				\vspace{2cm}
				%{\large \imprimirautor} 	
				\vfill
        {\large{\imprimirlocal~-~\imprimiruf \\ \imprimirano }}
        \end{center}
\end{capa}   } % Capa

%----------------------------------------------------
\renewcommand{\imprimirfolhaderosto}{% folha de rosto
    \begin{center}
    {{\MakeTextUppercase \imprimirautor}}  \\
		\vfill
		\large {\textbf{\MakeTextUppercase{\imprimirtitulo}}\\
        \MakeTextLowercase{\textbf{\imprimirsubtitulo}}}
    \end{center}
    \vfill 
    \begin{flushright} 
    \parbox{0.6\linewidth}{
		\imprimirtipotrabalho~ apresentado ao Curso de \imprimircurso~ do \imprimirinstituicao~ - \imprimirinstituto~ para a obtenção do título de \imprimirgrau. \\
		\vfill
		\textbf{\imprimirorientadorRotulo}~\imprimirorientador \\
		\vfill 
		\textbf{\imprimircoorientadorRotulo}~\imprimircoorientador}
   \end{flushright} 
   
	 \vfill
   \begin{center}
   {\large{\imprimirlocal~- \imprimiruf \\ \imprimirano}}
   \end{center} }  % folha de rosto

%----------------------------------------------------


%================================================================================
% Pacotes de citações
%================================================================================

% Configurações do pacote backref
% Texto padrão antes do número das páginas
\renewcommand{\backref}{}
% Define os textos da citação
\addto\captionsbrazil{% portugues-brasil
    % Usado sem a opção hyperpageref de backref
    \renewcommand{\backrefpagesname}{Citado na(s) p{\'a}gina(s):~}
    \renewcommand*{\backrefalt}[4]{
    	\ifcase #1 %
    		Nenhuma cita{\c c}{\~a}o no texto.%
    	\or
    		Citado na p{\'a}gina #2.%
    	\else
    		Citado #1 vezes nas p{\'a}ginas #2.%
    	\fi}%
}
\addto\captionsenglish{% ingles
    % Usado sem a opção hyperpageref de backref
    \renewcommand{\backrefpagesname}{Cited on page(s):~}
    \renewcommand*{\backrefalt}[4]{
    	\ifcase #1 %
    		No citation.%
    	\or
    		Cited on page #2.%
    	\else
    		Cited #1 times on pages #2.%
    	\fi}%
}
% ---
\makeatletter
\newcommand\thefontsize{Fonte tamanho: \f@size pt}
\makeatother



% para não deixar uma figura sozinha no centro de uma página,
% mas no topo da página.
\makeatletter
\setlength{\@fptop}{0pt}
\makeatother




%--  % Estrutura do Texto e Pacotes Principais

% -- Informações para Capa e Folha de Rosto a serem editadas

\titulo{DRD - Uma aplicação de Reconciliação de dados utilizando métodos de Lagrange} 
% caso o trabalho não tenha subtítulo comentar linha abaixo e retirar dois pontos do título do trabalho
% \subtitulo{Uma aplicação web de análise e reconciliação de dados utilizando métodos de Lagrange}

\autor{Nilton Aguiar dos Santos} \autorcite{Santos, Nilton}
\local{Várzea Grande} \uf{MT}
\data{31 de agosto de 2024} \dia{31} \mes{08} \ano{2024} %deixar sem preencher antes do TCC
\orientador{Prof. Dr. João Gustavo Coelho Pena}  % Nome do orientador 
\ttorientador{UFMT} % Instituição do orientador
\coorientador{Prof. Me. Nome do Coorientador}   % Nome do coorientador
\ttcoorientador{UFMT} % Instituição do Coorientador
\instituicao{Universidade Federal de Mato Grosso} \sigla{UFMT}
\instituto{Instituto de Engenharia do Campus de Várzea Grande}
\curso{Engenharia de Computação}	
\tipotrabalho{Trabalho de conclusão de curso}
\grau{Engenheiro de Computação}

%------Nomes dos examinadores.  
\examinadorum{Prof. Me. Membro da Banca 1} \ttexaminadorum{UFMT}
\examinadordois{Prof. Dr. Membro da Banca  2} \ttexaminadordois{UFMT}

% ------------------------------------------------------
\makeindex   

\begin{document} % Início do documento

\frenchspacing  % Retira espaço obsoleto entre as frases.

% ----------------------------------------------------------
% -- Elementos Pré-Textuais: -------------------------------
\pagenumbering{roman}

\imprimircapa  % Capa
\imprimirfolhaderosto % Folha de rosto

% ---------------------------------------------------------------
% ----------------  Ficha Catalográfica  -------------------------
% ---------------------------------------------------------------
% Modelo de ficha catalográfica. Você deverá substituir esta
% folha na versão final da monografia por um pdf fornecido pela 
% biblioteca. Salve o modelo oficial como ficha_catalografica.pdf
% e use o comando abaixo para inseri-lo na versão final do texto.

%\begin{fichacatalografica}
%    \includepdf{ficha_catalografica.pdf}
%\end{fichacatalografica}



%% Modelo de Como fazer a Ficha Catalográfica:

\begin{fichacatalografica}
	\sffamily
	\vspace*{\fill}					% Posição vertical
	\begin{center}					% Minipage Centralizado
	\fbox{\begin{minipage}[c][8cm]{13.5cm}		% Largura
	
	Ficha catalográfica a ser fornecida pela biblioteca.
% 	\small
% 	\imprimirautorcite.
% 	%Sobrenome, Nome do autor
	
% 	\hspace{0.5cm}  \\
% 	\imprimirtitulo  / \imprimirautor. --, \imprimirano-
	
% 	\hspace{0.5cm} \pageref{LastPage} p. 1 :il. (colors; grafs; tabs).\\
	
% 	\hspace{0.5cm} \imprimirorientadorRotulo~\imprimirorientador\\
	
% 	\hspace{0.5cm}
% 	\parbox[t]{\textwidth}{\imprimirtipotrabalho~--\\ \imprimirinstituicao. \\
% 	\imprimirinstituto. \\ ~\imprimirdepartamento.}\\
	
% 	\hspace{0.5cm}
% 		1. Palavra-chave1.
% 		2. Palavra-chave2.
% 		2. Palavra-chave3.
% 		I. \imprimirorientador.
% 		II. \imprimirinstituicao.
% 		III. Título 			
	\end{minipage}}
	\end{center}
\end{fichacatalografica}



% ---------------------------------------------------------------
% ----------------  Folha de aprovação  -------------------------
% ---------------------------------------------------------------
% Modelo de Folha de aprovação. Você deverá substituir esta
% folha na versão final da monografia por um pdf fornecido pelo  
% colegiado do seu curso. Salve o modelo oficial como 
% folhadeaprovacao_final.pdf e use o comando abaixo para
% inseri-lo na versão final do texto.

%\begin{fichacatalografica}
%    \includepdf{folhadeaprovacao_final.pdf}
%\end{fichacatalografica} Esta folha será 


\begin{folhadeaprovacao}

\begin{center}
    \large { \imprimirautor}  \\
		\vspace{3cm} 
		\large {\textbf{\MakeTextUppercase{\imprimirtitulo}}~\MakeTextLowercase{\imprimirsubtitulo}}
    \end{center}
    \vspace{2cm}
    \begin{flushright} 
    \parbox{0.6\linewidth}{
		\imprimirtipotrabalho~ apresentado ao Curso de \imprimircurso~ do \imprimirinstituicao~ - \imprimirinstituto~ para a obtenção do título de \imprimirgrau. \\
		}
   \end{flushright} 
  % \vspace{2cm}
   \vfill
 
   \begin{center}
   \large{
   Aprovado em:~\imprimirdia /\ \imprimirmes/\ \imprimirano~ pela banca examinadora:
    \vspace{2cm}
    \vfill
          \rule{15cm}{.1pt} \\
      {\imprimirorientador}~-~{\imprimirttorientador}~(Orientador) 
      \vfill
			 \ifdefvoid{\imprimircoorientador}{}{
      \rule{15cm}{.1pt} \\
      \imprimircoorientador~-~\imprimirttcoorientador~(Coorientador) }
			 \vfill
      \rule{15cm}{.1pt} \\
      {\imprimirexaminadorum}~-~{\imprimirttexaminadorum} %\\ Examinador
        \vfill
        \ifdefvoid{\imprimirexaminadordois}{}{
        \rule{15cm}{.1pt} \\
        \imprimirexaminadordois~-~\imprimirttexaminadordois %\\ Examinador
        }
				\vfill
        \ifdefvoid{\imprimirexaminadortres}{}{
        \rule{15cm}{.1pt} \\
        \imprimirexaminadortres~-~\imprimirttexaminadortres %\\ Examinador
        }
				\vfill
        \ifdefvoid{\imprimirexaminadorquatro}{}{
        \rule{15cm}{.1pt} \\
        \imprimirexaminadorquatro~-~\imprimirttexaminadorquatro %\\ Examinador
        }
   
   }
   \end{center} 

% \begin{center}
%     {\large \textsc{\imprimirautor}} \\
%   	\vspace{2cm}	
%     {\textsc{\Large \textbf{\imprimirtitulo}}} \\
% 		\vspace{2cm}
% \end{center}		

% \noindent \imprimirtipotrabalho~ defendido e aprovado em \imprimirlocal,~ \imprimirdata~,  pela banca examinadora constituída pelos professores:
% \vspace{2cm}
% \begin{center}

%       \rule{10cm}{.1pt} \\
%       {\imprimirorientador} \\ {\imprimirttorientador} \\
% 			 Orientador 
%       \vfill
% 			 \ifdefvoid{\imprimircoorientador}{}{
%       \rule{10cm}{.1pt} \\
%       \imprimircoorientador \\ \imprimirttcoorientador \\ Coorientador }
% 			 \vfill
%       \rule{10cm}{.1pt} \\
%       {\imprimirexaminadorum} \\ {\imprimirttexaminadorum} \\ Examinador
%         \vfill
%         \ifdefvoid{\imprimirexaminadordois}{}{
%         \rule{10cm}{.1pt} \\
%         \imprimirexaminadordois \\ \imprimirttexaminadordois \\ Examinador}
% 				\vfill
%         \ifdefvoid{\imprimirexaminadortres}{}{
%         \rule{10cm}{.1pt} \\
%         \imprimirexaminadortres \\ \imprimirttexaminadortres \\ Examinador}
% 				\vfill
%         \ifdefvoid{\imprimirexaminadorquatro}{}{
%         \rule{10cm}{.1pt} \\
%         \imprimirexaminadorquatro \\ \imprimirttexaminadorquatro \\ Examinador}
% \end{center}
  
\end{folhadeaprovacao}
% --- 

\begin{dedicatoria}
   \vspace*{\fill}
   \begin{flushright} 
        \parbox{0.6\linewidth}{
		 {
            Aos meus pais, José Roberto dos Santos e Teresa de Jesus de Souza Aguiar, por cada sacrifício, por cada gesto de amor, por cada exemplo de dedicação. Aos meus irmãos por acreditaram em meu potencial e caminharam ao meu lado nesta jornada acadêmica. À minha namorada por ter me motivado a cada segundo do processo.
		}
		}
   \end{flushright} 
   \vspace{2cm}
	 
\end{dedicatoria}
\begin{agradecimentos}

Agradeço à Faculdade de Engenharia da Universidade Federal de Mato Grosso pela excelência acadêmica proporcionada, que enriqueceu meu conhecimento e habilidades. 

Expresso minha gratidão à dedicada equipe de professores e funcionários, cujo comprometimento a frente de tantas dificuldades e complicações, não recuaram em oferecer um ensinamento completo a cada aluno, que contribuiu significativamente para o meu crescimento profissional. Esta instituição desempenhou um papel fundamental no meu desenvolvimento acadêmico, sendo um privilegiado ambiente de aprendizado.

Agradeço ao professor João Gustavo Coelho Pena pela confiança, parceria, pela atenção e confiança, que possibilitou a construção desse trabalho.
\end{agradecimentos}
\begin{epigrafe}
    \vspace*{\fill}
	\begin{flushright}
	    \parbox{0.6\linewidth}{
	% 	“Estude bastante o que mais lhe interessa da maneira mais indisciplinada, irreverente e original possível." \\
		
	% 	Richard Feynman
	% }
 		“Não é o conhecimento, mas o ato de aprender, não a posse, mas o ato de chegar lá, que garante o maior prazer." \\
		
		Carl Friedrich Gauss
	}
	\end{flushright}
	\vspace{2cm}
\end{epigrafe}
%--------------------------------------------------------------------------
%--------------------- Resumo em Português --------------------------------
%--------------------------------------------------------------------------

%\setlength{\absparsep}{18pt} % ajusta o espaçamento dos parágrafos do resumo
\begin{resumo}

Neste trabalho de conclusão de curso, desenvolveu-se DRD (Dashboard of Data Reconciliation), um software online de análise, reconciliação e qualidade de dados utilizando técnicas de minimização de funções multivariáveis pelo método dos multiplicadores de Lagrange. A solução toma como prioridade uma abordagem baseada na web, oferecendo ao usuário a capacidade de realizar a análise e reconciliação de dados de forma remota e eficiente, com foco nos conceitos computacionais modernos afim dessa experiência ser a mais facilitada possível. Se utiliza durante todo o software a aplicação dos cálculos matemáticos e estatísticos voltados ao problema de análise, reconciliação e qualidade de dados. 

Por meio da ferramenta é possível modelar todo um processo industrial e alimenta-lo com os dados oriundos da planta em questão e a partir disso analisar, reconciliar e verificar a qualidade dos dados em tempo real. O software é uma solução inovadora dado que não há um direto competidor para as funções oferecidas em um ambiente mais acessível e ágil no estado de Mato Grosso. Ao longo deste trabalho, o processo filosófico do desenvolvimento, os cálculos matemáticos, os conceitos estatísticos, computacionais e a lógica do software aplicado como solução são explicados de forma detalhada. Exemplos do código funcional e considerações finais sobre o trabalho realizado são apresentados nos últimos capítulos.

 \vspace{\onelineskip}
 \noindent
 \textbf{Palavras-chave}: Reconciliação de Dados. Análise de Qualidade de Dados. Desenvolvimento Web. Multiplicadores de Lagrange. Desenvolvimento de Software. 
\end{resumo}

%--------------------------------------------------------------------------
%--------------------- Resumo em Inglês --------------------------------
%--------------------------------------------------------------------------
\begin{resumo}[\large ABSTRACT]
\begin{otherlanguage*}{english}

In this undergraduate thesis, the development of DRD (Dashboard of Data Reconciliation) was undertaken, an online software for data analysis, reconciliation, and quality assessment using techniques of multivariable function minimization through the method of Lagrange multipliers. The solution prioritizes a web-based approach, providing users with the ability to remotely and efficiently perform data analysis and reconciliation, focusing on modern computational concepts to ensure a user-friendly experience. The software applies mathematical and statistical calculations throughout, specifically tailored to the problems of data analysis, reconciliation, and quality.

Through this tool, it is possible to model an entire industrial process and feed it with data from the relevant plant, allowing real-time analysis, reconciliation, and quality verification. The software stands out as an innovative solution, as there is no direct competitor offering similar functions in a more accessible and agile environment in the state of Mato Grosso. This work details the philosophical process of development, mathematical calculations, statistical and computational concepts, and the logic of the software as an applied solution. Examples of functional code and final considerations about the work are presented in the last chapters.

\vspace{\onelineskip}
\noindent 
\textbf{Keywords}: Data Reconciliation. Data Quality Analysis. Web Development. Lagrange Multipliers. Software Development.
\end{otherlanguage*}
\end{resumo}

\renewcommand{\listfigurename}{\large LISTA DE ILUSTRA\c{C}\~{O}ES}
\pdfbookmark[0]{\listfigurename}{lof}
\listoffigures*   % Cria a Lista de Figuras
\cleardoublepage

% inserir lista de quadros
% ---
\pdfbookmark[0]{\listofquadrosname}{loq}
\listofquadros*
\cleardoublepage
% ---

\pdfbookmark[0]{\listtablename}{lot}
\listoftables*  % Cria a lista de Tabelas
\cleardoublepage

%\renewcommand{\listalgorithmcfname}{Lista de algoritmos}
%\pdfbookmark[0]{\listalgorithmcfname}{lof}
%\listofalgorithmes   % Cria a lista de Tabelas
%\cleardoublepage

% ---------------------------------------------------
% ------ Lista de abreviaturas e siglas -------------
% ---------------------------------------------------
\begin{siglas}
  \item[ABNT] Associação Brasileira de Normas Técnicas
  \item[SEP] Sistema 
  \item[IFMG] Instituto Federal de Minas Gerais
\end{siglas}
% ---------------------------------------------------
% ----------- Lista de símbolos ---------------------
% ---------------------------------------------------

\begin{simbolos}
  \item[$ \Gamma $] Letra grega Gama
  \item[$ \Lambda $] Lambda
  \item[$ \zeta $] Letra grega minúscula zeta
  \item[$ \xi$] Letra grega minúscula qsi
  \item[$ \in $] Pertence
\end{simbolos}


\pdfbookmark[0]{\contentsname}{toc}
\tableofcontents*

\cleardoublepage




% ----------------------------------------------------------
% -- Capítulos do Trabalho: --------------------------------
\pagenumbering{arabic} 
\setcounter{page}{14} % Inserir aqui o número de páginas antes da introdução menos a capa
\textual 

%%%%%%%%%%%% ORDEM DOS ARQUIVOS %%%%%%%%%%%%%%%%%%%

\chapter{Introdução} \label{Introducao}

O panorama atual sugere que o setor industrial brasileiro continua a crescer (Tomás Torezani, 2020) porém com ele deve seguir paralelamente um avanço tecnológico para que ela consiga aumentar o nível de produtividade dessas indústrias (Marcos Lisboa, 2021), dessa forma se origina o funcionalidade do trabalho de conclusão de curso DRD (Dashboard de Reconciliação de Dados), na qual ele unifica a área de análise de qualidade dos dados e a reconciliação desses mesmos com a tecnologia oriunda dos avanços tecnológicos na área de desenvolvimento web.

\section{Objetivos}

\subsection{\textit{Objetivo geral}}

Considerado a necessidade de inovação na área industrial e da união entre essas duas áreas, o objetivo deste trabalho é apresentar o desenvolvimento de um sistema online de reconciliação de dados utilizando métodos matemáticos criados por Lagrange.

\subsection{\textit{Objetivos específicos}}

Para alcançar o objetivo proposto neste trabalho, cuja finalidade é desenvolver uma ferramenta online de reconciliação de dados é necessário especificar os seguintes objetivos específicos:

a) Realizar a pesquisa de estado da arte da área de desenvolvimento de software para tal função.

b) Analisar e compreender as metodologias e algoritmos existentes na área de reconciliação de dados, destacando suas aplicações, vantagens e limitações. A pesquisa de estado da arte é crucial para identificar as tendências mais recentes, as melhores práticas e as lacunas existentes, proporcionando uma base sólida para o desenvolvimento da ferramenta online.

c) Selecionar e adaptar as tecnologias adequadas para a implementação da ferramenta online de reconciliação de dados. Isso envolverá a avaliação de linguagens de programação, frameworks e plataformas que melhor se adéquem aos requisitos específicos da aplicação, considerando aspectos como eficiência, escalabilidade e segurança.

d) Projetar a arquitetura da ferramenta, delineando os componentes principais, a interação entre eles e a lógica de funcionamento. A clareza na definição da arquitetura será essencial para garantir a eficácia operacional da ferramenta, bem como para facilitar futuras atualizações e expansões.

e) Desenvolver a ferramenta online de reconciliação de dados, implementando os algoritmos e funcionalidades identificados na pesquisa de estado da arte. Durante essa fase, é crucial garantir a usabilidade, a integridade dos dados e a eficiência do sistema, atendendo aos padrões de qualidade e performance esperados.

f) Realizar testes abrangentes para validar a eficácia e confiabilidade da ferramenta. Isso incluirá testes de integração, testes de segurança e simulações de casos práticos para verificar a capacidade da ferramenta em lidar com diferentes cenários e volumes de dados.

g) Elaborar uma documentação completa, abrangendo desde o processo de desenvolvimento até as instruções de uso da ferramenta. Uma documentação robusta e clara será essencial para facilitar a compreensão, manutenção e futuras implementações relacionadas à ferramenta de reconciliação de dados.

\section{Justificativa}

A ferramenta online de reconciliação de dados baseia-se na interseção entre a indústria e tecnologia da internet 4.0. Observa-se a necessidade premente de ferramentas especializadas que possam acompanhar e potencializar essa convergência, contribuindo para a eficiência operacional, qualidade dos processos industriais e aprimoramento da competitividade das empresas. A atual revolução industrial demanda soluções tecnológicas avançadas que permitam a integração e análise eficiente de dados em tempo real. A solução DRD torna-se crucial para manter a integridade das informações, garantindo a consistência e confiabilidade necessárias para operações industriais avançadas.

A utilização de uma ferramenta online de reconciliação de dados abre caminho para a otimização de processos, unificação com outras ferramentas, redução de erros, monitoramento em tempo real e, consequentemente, aumento da eficiência operacional. Isso não apenas reduz custos operacionais, mas também posiciona as empresas em um patamar competitivo mais vantajoso. E a integração dessa ferramenta com outras permite não apenas a reconciliação de dados, mas também a extração de insights valiosos para a tomada de decisões estratégicas. Essa sinergia contribui para a inovação contínua e a adaptação rápida às mudanças no ambiente industrial.

O desenvolvimento de uma ferramenta posiciona-se em consonância com as tendências globais de transformação digital e automação industrial. Isso não apenas fortalece a competitividade das empresas no mercado nacional, mas também as prepara para atuar de maneira efetiva em uma escala internacional. Em suma, diante do cenário tecnológico atual e das demandas evolutivas da indústria, a criação de uma ferramenta online de reconciliação de dados emerge como uma resposta estratégica para potencializar a interseção entre a Internet e o setor industrial, impulsionando a eficiência e a inovação nas práticas industriais contemporâneas.

\section{Estado da Arte}

Este trabalho está organizado da seguinte forma: (descrever)....

\section{Organização do Texto}

Este trabalho está organizado da seguinte forma: (descrever)....

%Sugestões para estrutura da monografia:

% \begin{enumerate}[label=(\alph*)]
%   \item Introdução
%   \item Referencial Teórico (ou Revisão Bibliográfica)
%   \item Materiais e Métodos (ou Metodologia)
%   \item Resultados e Discussões
%   \item Conclusão (ou Considerações Finais)
%   \item Referências
% \end{enumerate}
\chapter{Referencial Teórico} \label{RevisaoBibliografica}

\section{Reconciliação de Dados no Âmbito Industrial}
% Reconciliação de Dados
% ========================================================
% QUESTÃO:

Já no começo da década de 60 se entendeu a importância do controle de processos químicos industriais por computadores utilizando técnicas matemáticas (KUEHN, 1961), dessa forma surge a área da computação voltada à reconciliação de dados, na qual há a comparação, validação e correção de informações coletadas em diferentes pontos do processo, afim de determinar a consistência dos dados, a qualidade dos mesmos ou até otimizar os processos (NARASIMHAN, 2000).

Ao longo das últimas décadas, os métodos de reconciliação de dados evoluíram significativamente, acompanhando os avanços tecnológicos e as demandas crescentes da indústria [REFERÊNCIA]. Com o advento de sistemas de automação mais avançados, sensores inteligentes e a proliferação de dispositivos conectados, a quantidade de dados gerados nas operações industriais aumentou drasticamente [REFERÊNCIA]. Nesse contexto, a reconciliação, análise e gestão de dados tornaram-se ferramentas indispensáveis para garantir a integridade e a confiabilidade das informações coletadas em tempo real[REFERÊNCIA].

Na era contemporânea, a reconciliação de dados desempenha um papel crucial na otimização dos processos industriais, contribuindo para a eficiência operacional e a redução e custos [REFERÊNCIA]. Sistemas avançados de reconciliação não apenas comparam e validam dados, mas também utilizam algoritmos sofisticados para identificar padrões, tendências e anomalias [REFERÊNCIA]. Essa capacidade analítica permite que as indústrias ajam proativamente, antecipando-se a problemas potenciais, otimizando a produção e melhorando a qualidade dos produtos finais [REFERÊNCIA].

\section{Utilização do Método dos Multiplicadores de Lagrange}
% ---
% Reconciliação de dados utilizando o método dos multiplicadores de Lagrange
% ========================================================
% QUESTÃO: Será que é necessário uma maior explicação sobre esse tópico?

No sistema em questão a reconciliação de dados vai ser feita com a minimização de funções multivariáveis utilizando método dos multiplicadores de Lagrange, desenvolvida pelo matemático Joseph Louis Lagrange (1739-1813), que desenvolveu um método de encontrar o mínimo ou máximo de uma função multivariável sujeita a uma ou várias condições de restrição.

Nesse contexto, a aplicação do método dos multiplicadores de Lagrange destaca-se como uma abordagem matemática sofisticada e eficaz para resolver problemas complexos de reconciliação de dados. A técnica proporciona uma estrutura robusta para lidar com situações em que é necessário otimizar uma função multivariável sujeita a restrições específicas. Ao utilizar os multiplicadores de Lagrange, o sistema ganha a capacidade de encontrar soluções que atendam simultaneamente às condições impostas, proporcionando uma reconciliação precisa e eficiente dos dados envolvidos. Essa metodologia, fundamentada em princípios matemáticos sólidos, contribui para aprimorar a qualidade e a confiabilidade dos resultados obtidos, na qual a torna uma ferramenta valiosa para a solução proposta no trabalho.

\section{Desenvolvimento de uma Ferramenta Online}
% Desenvolvimento de uma ferramenta online
% ========================================================
% QUESTÃO: 

Vestibulum mollis pulvinar venenatis. Morbi maximus interdum ipsum ac tincidunt. Cras vulputate volutpat ex, eu consequat tellus scelerisque a. Integer a commodo purus, eleifend suscipit nibh. Nulla vitae varius mauris. In hac habitasse platea dictumst. Nullam euismod et nulla nec aliquet. Nam placerat faucibus quam ac vehicula. Etiam vel ex massa. Donec faucibus, dolor sed eleifend gravida, nisl urna facilisis dui, et volutpat turpis nisi eu augue. Aliquam egestas purus at nulla bibendum efficitur. Nam mollis vitae arcu a interdum. In vel gravida nunc. Nam eu risus ut odio suscipit varius. Nunc vehicula, augue tempor congue accumsan, metus felis commodo dolor, a pharetra risus elit in erat.

Sed eros urna, accumsan eget malesuada at, vestibulum at turpis. Nullam hendrerit justo orci, et elementum neque pulvinar sed. Mauris odio nunc, fermentum scelerisque elit vitae, ornare varius turpis. Integer id aliquam urna. Mauris diam arcu, cursus sit amet augue id, pretium mollis risus. Nulla facilisi. Suspendisse id erat velit. Mauris suscipit tempor metus, vel venenatis velit malesuada ac. Donec vel eros libero.

\section{A Sinergia Entre a Indústria e a Internet}
% A sinergia entre a indústria e a internet
% ========================================================
% QUESTÃO: 

O panorama atual de avanço da internet e a convergência entre a internet e o setor industrial representam um arco significativo na era da Engenharia de Computação [REFERÊNCIA]. Este fenômeno transformador tem sido impulsionado pela fusão das tecnologias da informação com os processos industriais, dando origem a conceitos como Indústria 4.0. 

No âmbito desta ferramenta é aplicado a intersecção dessas duas esferas, onde os conceitos de práticas industriais, reconciliação, análise e qualidade de dados se integram à internet na qual é extraído deles o seu maior forte, como uma maior integralidade com outros sistemas por meio de APIs (interfaces de programação de aplicativos), melhor interatividade entre os elementos do sistema, promovendo uma comunicação mais dinâmica e eficaz, aumento da eficiência operacional e facilidade na gestão de processos [REFERÊNCIA]. Essa sinergia possibilita a criação de ecossistemas industriais mais conectados nos quais os dados relevantes podem ser tratados de forma segura e eficiente. [REFERÊNCIA].

O horizonte atual, delineado pelos recentes avanços tecnológicos e inovações sustenta a perspectiva otimista que as indústrias estão destinadas a experimentar um crescimento substancial no país. A convergência entre a internet e o setor industrial, representa um catalisador significativo para a modernização e eficiência operacional. A integração de práticas avançadas de desenvolvimento de soluções voltadas a usabilidade e ambiente de desenvolvimento com controle computacional, como a reconciliação de dados e análise preditiva, impulsiona a qualidade e consistência dos processos produtivos.

Além disso, a aplicação da internet nas práticas industriais não só fortalece a competitividade das empresas mas também desempenha um papel crucial na expansão econômica do país. A capacidade de adotar tecnologias inovadoras como a automação avançada, coloca as indústrias em uma posição estratégica para atender às crescentes demandas do mercado e elevar a produtividade [REFERÊNCIA]. Nesse sentido é plausível afirmar que diante do atual cenário tecnológico e das tendências emergentes, é indubitável a necessidade e importância da ferramenta desenvolvida nesse trabalho.
\chapter{Metodologia} \label{metodologia}

\section{Sessão 1}
% 
% ========================================================
% QUESTÃO:

Vivamus vitae arcu ullamcorper, eleifend leo vel, aliquet orci. Aenean pellentesque turpis quis mi vehicula bibendum. In tempor leo nunc, eu pellentesque purus euismod ac. Proin id massa libero. Duis ligula libero, scelerisque id diam sit amet, mollis pharetra urna. Sed vehicula nibh et metus viverra fermentum. Pellentesque tortor sem, elementum sit amet nunc ac, dapibus vulputate tellus. Vestibulum eleifend a lorem finibus euismod. Nulla vulputate odio quis velit congue eleifend ut quis quam.


\subsection{Subsessão 1}

Maecenas sit amet tortor neque. Nunc et tristique justo. Pellentesque eget scelerisque eros, vitae aliquet tortor. Aliquam ut tortor tempor, euismod massa in, interdum nisi. Aliquam congue congue libero, a faucibus magna cursus id. Aenean ut diam sit amet dolor tempor placerat ut a ex. Donec egestas est at nulla ultricies, quis mollis tortor ultricies. Sed et consequat enim. Nam non consectetur metus. Cras magna sem, laoreet gravida mauris id, sollicitudin tristique libero. Mauris sodales vel nisl ac vestibulum.

\clearpage

\chapter{Resultados} \label{resultado}

\section{Sessão 1}
% 
% ========================================================
% QUESTÃO:

Vivamus vitae arcu ullamcorper, eleifend leo vel, aliquet orci. Aenean pellentesque turpis quis mi vehicula bibendum. In tempor leo nunc, eu pellentesque purus euismod ac. Proin id massa libero. Duis ligula libero, scelerisque id diam sit amet, mollis pharetra urna. Sed vehicula nibh et metus viverra fermentum. Pellentesque tortor sem, elementum sit amet nunc ac, dapibus vulputate tellus. Vestibulum eleifend a lorem finibus euismod. Nulla vulputate odio quis velit congue eleifend ut quis quam.


\subsection{Subsessão 1}

Maecenas sit amet tortor neque. Nunc et tristique justo. Pellentesque eget scelerisque eros, vitae aliquet tortor. Aliquam ut tortor tempor, euismod massa in, interdum nisi. Aliquam congue congue libero, a faucibus magna cursus id. Aenean ut diam sit amet dolor tempor placerat ut a ex. Donec egestas est at nulla ultricies, quis mollis tortor ultricies. Sed et consequat enim. Nam non consectetur metus. Cras magna sem, laoreet gravida mauris id, sollicitudin tristique libero. Mauris sodales vel nisl ac vestibulum.

\clearpage

%\chapter{Considerações Finais} \label{consideracoes}

%%====== Section ========%
\chapter{Conclusão e Trabalhos Futuros}\label{conclusao}
\section{Reconciliação de Dados no Âmbito Industrial}
% Reconciliação de Dados
% ========================================================
% QUESTÃO:

Já no começo da década de 60 se entendeu a importância do controle de processos químicos industriais por computadores utilizando técnicas matemáticas (KUEHN, 1961), dessa forma surge a área da computação voltada à reconciliação de dados, na qual há a comparação, validação e correção de informações coletadas em diferentes pontos do processo, afim de determinar a consistência dos dados, a qualidade dos mesmos ou até otimizar os processos (NARASIMHAN, 2000).




% ----------------------------------------------------------
% -- Elementos Pós-Textuais: -------------------------------
\postextual  
\bibliography{Bibliografia/bibliografia} % Referências bibliográficas
%-------------------------------------------------------------
%---------------------- Apêndices ----------------------------
%-------------------------------------------------------------

\begin{apendicesenv}
%\partapendices  % Indica o início dos Apendices
\chapter{Informações para complementar o texto}

Note que os Apêndices são dedicados aos textos ou \textbf{documentação elaborados pelo próprio autor} que complemente  a  argumentação  textual  (códigos,  reportagens,  relatórios  etc.).

\lipsum[8]



\end{apendicesenv}
% ----------------------------------------------------------
\chapter{Descrição}
% ----------------------------------------------------------

São documentos não elaborados pelo autor que servem como fundamentação (mapas, leis, estatutos). Deve ser precedido da palavra ANEXO, identificada por letras maiúsculas consecutivas, travessão e pelo respectivo título. Utilizam-se letras maiúsculas dobradas quando esgotadas as letras do alfabeto. 

%\includepdf[pages={1,3-5}]{PosTextuais/includepdfpages.pdf}
%\phantompart  \printindex  % Indice Remissivo
% ----------------------------------------------------------
\end{document}  % fim do documento
