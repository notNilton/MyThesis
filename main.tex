%%   eq:     Engenharia Química
%%   em:     Engenharia de Minas
%%   ec:     Engenharia de Computação
%%   et:     Engenharia de Transportes
%%   eca:    Engenharia de Controle e Automação
%% ------------------------------
\documentclass[tcc/eca]{faeng}
%% ==============================

%% ==============================
%% PACOTES
%% ------------------------------
%% Pacotes fundamentais 
%% ------------------------------
\usepackage{cmap}			% Mapear caracteres especiais no PDF
\usepackage{lmodern}		% Usa a fonte Latin Modern			
\usepackage{makeidx}        % Cria o indice
\usepackage{hyperref}  		% Controla a formação do índice
\usepackage{lastpage}		% Usado pela Ficha catalográfica
\usepackage{indentfirst}	% Indenta o primeiro parágrafo de cada seção.
\usepackage{nomencl} 		% Lista de simbolos
\usepackage{graphicx}		% Inclusão de gráficos
\usepackage[brazil]{babel}  % Codificação e uso de caracteres especiais.
\usepackage[utf8]{inputenc}
%% ------------------------------
%% Pacotes adicionais, usados apenas no âmbito do Modelo faeng
%% ------------------------------
\usepackage{lipsum}				       % para geração de dummy text
\usepackage[printonlyused]{acronym}
\usepackage{xcolor}
\usepackage{booktabs}
\usepackage{multirow}
%% ==============================

%% ==============================
%% Informações de dados para CAPA e FOLHA DE ROSTO
%% ------------------------------
%% Título:
%%	1. Título em português
%%	2. Título em inglês
\titulo{DRD - Uma aplicação web de análise e reconciliação de dados utilizando métodos de Lagrange}{DRD - A data analysis and reconciliation dashboard web application using Lagrange's methods}

% Possiveis outras escolhas para o titulo.

% \titulo{DRD - Aplicação Web de Reconciliação de Dados utilizando o método de minimização de funções multivariáveis com multiplicadores de Lagrange}{DRD - Data Reconciliation Dashboard web application using the multivariable function minimization method with Lagrange multipliers}

% Aplicação web de dashboard de reconciliação de dados utilizando o método de minimização de funções multivariáveis com multiplicadores de Lagrange
%% ------------------------------
%% Autor:
%%	1. Nome completo do autor
%%	2. Formato de nome para bibliografia
\autor{Nilton Aguiar dos Santos}{Santos, Nilton}
%% ------------------------------
%% Cidade
\local{Várzea Grande}
%% ------------------------------
%% Ano de defesa
\data{2024}
%% ------------------------------
%% Nome do orientador
\orientador{João Gustavo Coelho Pena}
%% Nome do coorientador
%\coorientador{Nome completo do coorientador}
%% ==============================

%% ==============================
%% compila o indice
%% ------------------------------
\makeindex
%% ==============================

%% ==============================
%% Compila a lista de abreviaturas e siglas
%% ------------------------------
\makenomenclature
%% ==============================

%% ==============================
%% Inserir ficha catalográfica
%%
%% Caso o comando \inserirfichacatalografica seja definido, a ficha catalográfica
%% será inserida atrás da folha de rosto. Caso contrário a página será deixada em branco.
%%
%% CUIDADO: Esta opção deve ser preenchida antes do comando \maketitle
%% ------------------------------
%\inserirfichacatalografica{fichaCatalografica.pdf}
%% ==============================

%% ==============================
%% Inserir folha de aprovação
%%
%% Caso o comando \inserirfolhaaprovacao seja definido, a a folha de aprovação
%% será inserida.
%% CUIDADO: Esta opção deve ser preenchida antes do comando \maketitle
%% ------------------------------
%\inserirfolhaaprovacao{folhaAprovacao.pdf}
%% ==============================

%% ==============================
%% INÍCIO DO DOCUMENTO
%% ------------------------------
\begin{document}
%% ------------------------------
%% ELEMENTOS PRÉ-TEXTUAIS
%% ------------------------------
\pretextual
%% ------------------------------
%% Insere Capa, Folha de rosto, Ficha catalográfica (se inserida)
%% e folha de aprovação (se inserida).
%% ------------------------------
\maketitle
%% ------------------------------
%% Dedicatória
%% ------------------------------
\imprimirdedicatoria{Aos meus pais, José Roberto dos Santos e Teresa de Jesus de Souza Aguiar, por cada sacrifício, por cada gesto de amor, por cada exemplo de dedicação. Aos meus irmãos por acreditaram em meu potencial e caminharam ao meu lado nesta jornada acadêmica. À minha namorada por ter me motivado a cada segundo do processo.}
%% ------------------------------
%% Agradecimentos
%% ------------------------------
\imprimiragradecimentos{
Agradeço à Faculdade de Engenharia da Universidade Federal de Mato Grosso pela excelência acadêmica proporcionada, que enriqueceu meu conhecimento e habilidades. 

Expresso minha gratidão à dedicada equipe de professores e funcionários, cujo comprometimento a frente de tantas dificuldades e complicações, não recuaram em oferecer um ensinamento completo a cada aluno, que contribuiu significativamente para o meu crescimento profissional. Esta instituição desempenhou um papel fundamental no meu desenvolvimento acadêmico, sendo um privilegiado ambiente de aprendizado.

Agradeço ao professor João Gustavo Coelho Pena pela confiança, parceria, pela atenção e confiança, que possibilitou a construção desse trabalho.
}
%% ------------------------------
%% Epígrafe
%% ------------------------------
\imprimirepigrafe{
		``Estude bastante o que mais lhe interessa da maneira mais indisciplinada, irreverente e original possível.''\\
		(Richard Feynman)
}
%% ==============================

%% ==============================
%% RESUMO e ABSTRACT
%% ------------------------------
%% Resumo em português
%% ------------------------------ 
\begin{resumo}{Reconciliação de Dados. Análise de Qualidade de Dados. Desenvolvimento Web. Multiplicadores de Lagrange. Desenvolvimento de Software.}

Neste trabalho de conclusão de curso, desenvolveu-se DRD (Dashboard of Data Reconciliation), um software online de análise, reconciliação e qualidade de dados utilizando técnicas de minimização de funções multivariáveis pelo método dos multiplicadores de Lagrange. A solução toma como prioridade uma abordagem baseada na web, oferecendo ao usuário a capacidade de realizar a análise e reconciliação de dados de forma remota e eficiente, com foco nos conceitos computacionais modernos afim dessa experiência ser a mais facilitada possível. Se utiliza durante todo o software a aplicação dos cálculos matemáticos e estatísticos voltados ao problema de análise, reconciliação e qualidade de dados. 

Por meio da ferramenta é possível modelar todo um processo industrial e alimenta-lo com os dados oriundos da planta em questão e a partir disso analisar, reconciliar e verificar a qualidade dos dados em tempo real. O software é uma solução inovadora dado que não há um direto competidor para as funções oferecidas em um ambiente mais acessível e ágil no estado de Mato Grosso. Ao longo deste trabalho, o processo filosófico do desenvolvimento, os cálculos matemáticos, os conceitos estatísticos, computacionais e a lógica do software aplicado como solução são explicados de forma detalhada. Exemplos do código funcional e considerações finais sobre o trabalho realizado são apresentados nos últimos capítulos.

\end{resumo}
%% ------------------------------
%% Resumo em inglês
%% ------------------------------
\begin{abstract}{Data Reconciliation. Data Quality Analysis. Web Development. Lagrange Multipliers. Software Development}


[NÃO ATUALIZADO]

In this undergraduate thesis, an online data analysis, reconciliation, and quality software was developed, nameed DRD (Dashboard of Data Reconciliation) using techniques of multivariable function minimization through the method of Lagrange multipliers. The solution prioritizes a web-based approach, providing users with the ability to perform data analysis and reconciliation remotely and efficiently, focusing on modern computational concepts to enhance the user experience. Throughout the software, mathematical and statistical calculations are applied to address the issues of data analysis, reconciliation, and quality.

The tool allows modeling of an entire industrial process and feeds it with data from the relevant plant, enabling real-time analysis, reconciliation, and verification of data quality. The software stands as an innovative solution, as there is no direct competitor for the functions offered in a more accessible and agile environment in the state of Mato Grosso. Throughout this work, the philosophical development process, mathematical calculations, statistical and computational concepts, and the logic of the software as a solution are explained in detail. Examples of functional code and final considerations about the work are presented in the last chapters.
	
\end{abstract}
%% ==============================

%% ==============================
%% inserir lista de ilustrações
%% ------------------------------
\listailustracoes
%% ==============================

%% ==============================
%% inserir lista de tabelas
%% ------------------------------
\listatabelas
%% ==============================

%% ==============================
%% inserir lista de abreviaturas e siglas
%% ------------------------------
\listasiglas{abreviaturas}
%% ==============================

%% ==============================
%% inserir o sumario
%% ------------------------------
\sumario
%% ==============================

%% ==============================
%% ELEMENTOS TEXTUAIS
%% ------------------------------
\mainmatter
%% ==============================

%% ==============================
%% Capitulos 
%% ------------------------------
\chapter[Introdução]{Introdução}

% ---
\section{Visão Geral | Um problema de três vanguardas}
% ---

Já no começo da década de 60 se entendeu a importância do controle de processos químicos industriais por computadores utilizando técnicas matemáticas (KUEHN, 1961), dessa forma surge a área da computação voltada à reconciliação de dados, na qual há a comparação, validação e correção de informações coletadas em diferentes pontos do processo, afim de determinar a consistência dos dados, a qualidade dos mesmos ou até otimizar os processos (NARASIMHAN, 2000).

Ao longo das últimas décadas, os métodos de reconciliação de dados evoluíram significativamente, acompanhando os avanços tecnológicos e as demandas crescentes da indústria [REFERÊNCIA]. Com o advento de sistemas de automação mais avançados, sensores inteligentes e a proliferação de dispositivos conectados, a quantidade de dados gerados nas operações industriais aumentou drasticamente [REFERÊNCIA]. Nesse contexto, a reconciliação, análise e gestão de dados tornaram-se ferramentas indispensáveis para garantir a integridade e a confiabilidade das informações coletadas em tempo real[REFERÊNCIA].

Na era contemporânea, a reconciliação de dados desempenha um papel crucia na otimização dos processos industriais, contribuindo para a eficiência operacional e a redução e custos [REFERÊNCIA]. Sistemas avançados de reconciliação não apenas comparam e validam dados mas também utilizam algoritmos sofisticados para identificar padrões, tendências e anomalias [REFERÊNCIA]. Essa capacidade analítica permite que as indústrias ajam proativamente, antecipando-se a problemas potenciais, otimizando a produção e melhorando a qualidade dos produtos finais [REFERÊNCIA].

Da mesma forma é possível observar o crescimento do setor da Indústria 4.0 q

-> O panorama atual sugere que as industrias vão crescer no país, ou seja.

O panorama atual sugere que o setor químico industrial está passando por um crescimento nos últimos anos 
mas que é necessário um avanço tecnológico par, o que está faltando. [REFERÊNCIA]

-> Agora vem a parte da área tecnológica importante, 

-> Precisa-se investir nessa área, como diz tal artigo.




da utilização de técnicas computacionais em um processo industrial afim de ter o controle em relação a qualidade e análise dos dados de forma automatizada e metódica. Surge juntamente com isso a idéia de utilizar conceitos de reconciliação de dados, uma técnica, como também da análise do mesmo (KUEHN, 1960). [AUMENTAR A QUANTIDADE DE BACKGROUND]
Embora seja antiga essa 

[Existem muitas indústrias altamente tecnologicas sendo feitas no estado de Mato Grosso, com isso surge um vácuo em algumas áreas ainda não inteiramente desenvolvidas]
Com o advento de ferramentas construídas inteiramente na base web, surge um espaço

A 
-> Contar a História de Análise e Reconciliação de Dados

-> Contar a História Laplace

-> Contar a História de Aplicação Web

- Existe um acréscimo na construção de industrias.


- Panorama atual sugere que está acontecendo algo X. Esse comportamento X influencia diretamente Y. Neste sentido, é importante trabalhar em Y.


% ---
\section{Justificativas}
% ---


% ---
\section{Objetivos}
% ---

-> Falar o que é uma Reconciliação de Dados.

-> Falar o método matemático utilizado.

-> Falar que vamos atacar esse problemas de duas frentes, a frente matemática e a frente computacional.

Este documento e seu código-fonte são exemplos de referência de uso da classe \textsf{faeng.cls} (baseada na classe \textsf{eesc.cls}) e do pacote \textsf{abntex2cite}. O documento exemplifica a elaboração de trabalho acadêmico produzido conforme a \ac{ABNT} \ac{NBR} 14724:2011 \emph{Informação e documentação - Trabalhos acadêmicos - Apresentação}.

Este modelo é uma implementação das normas para produção de textos estabelecida pela Faculdade de Engenharia da Universidade Federal de Mato Grosso, Campus Universitário de Várzea Grande.

\chapter[Fundamentação Teórica]{Fundamentação Teórica}
% ---
% Introdução à Fundamentação Teórica
% ========================================================
% QUESTÃO:

Este trabalho apresenta o processo de desenvolvimento do sistema DRD, que é capaz de modelar uma planta industrial, analisar a da qualidade dos dados introduzidos e também a reconciliação desses mesmos dados. A utilização dessa ferramenta não só é capaz de interagir uma interface de programação de aplicação, possibilitando integração com outras ferramentas, mas como também permite acesso a uma interface intuitiva via navegador web na qual os usuários têm não apenas acesso fácil, mas também a capacidade de interagir dinamicamente com os dados, elevando a experiência de consulta a um patamar mais avançado e produtivo. 


% ---
\section{Da Reconciliação de Dados}
% ---
% Reconciliação de Dados
% ========================================================
% QUESTÃO:

Já no começo da década de 60 se entendeu a importância do controle de processos químicos industriais por computadores utilizando técnicas matemáticas (KUEHN, 1961), dessa forma surge a área da computação voltada à reconciliação de dados, na qual há a comparação, validação e correção de informações coletadas em diferentes pontos do processo, afim de determinar a consistência dos dados, a qualidade dos mesmos ou até otimizar os processos (NARASIMHAN, 2000).

Ao longo das últimas décadas, os métodos de reconciliação de dados evoluíram significativamente, acompanhando os avanços tecnológicos e as demandas crescentes da indústria [REFERÊNCIA]. Com o advento de sistemas de automação mais avançados, sensores inteligentes e a proliferação de dispositivos conectados, a quantidade de dados gerados nas operações industriais aumentou drasticamente [REFERÊNCIA]. Nesse contexto, a reconciliação, análise e gestão de dados tornaram-se ferramentas indispensáveis para garantir a integridade e a confiabilidade das informações coletadas em tempo real[REFERÊNCIA].

Na era contemporânea, a reconciliação de dados desempenha um papel crucia na otimização dos processos industriais, contribuindo para a eficiência operacional e a redução e custos [REFERÊNCIA]. Sistemas avançados de reconciliação não apenas comparam e validam dados mas também utilizam algoritmos sofisticados para identificar padrões, tendências e anomalias [REFERÊNCIA]. Essa capacidade analítica permite que as indústrias ajam proativamente, antecipando-se a problemas potenciais, otimizando a produção e melhorando a qualidade dos produtos finais [REFERÊNCIA].

% Filosofia matemática por trás da minimização de funções multivariáveis utilizando o método dos multiplicadores de Lagrange.
% ========================================================
% QUESTÃO: Será que é necessário uma maior explicação sobre esse tópico?

No sistema em questão a reconciliação de dados vai ser feita com a minimização de funções multivariáveis utilizando método dos multiplicadores de Lagrange, desenvolvida pelo matemático Joseph Louis Lagrange (1739-1813), na qual desenvolveu um método de encontrar o mínimo ou máximo de uma função multivariável sujeita a uma ou várias condições de restrição.

Nesse contexto, a aplicação do método dos multiplicadores de Lagrange destaca-se como uma abordagem matemática sofisticada e eficaz para resolver problemas complexos de reconciliação de dados. A técnica proporciona uma estrutura robusta para lidar com situações em que é necessário otimizar uma função multivariável sujeita a restrições específicas. Ao utilizar os multiplicadores de Lagrange, o sistema ganha a capacidade de encontrar soluções que atendam simultaneamente às condições impostas, proporcionando uma reconciliação precisa e eficiente dos dados envolvidos. Essa metodologia, fundamentada em princípios matemáticos sólidos, contribui para aprimorar a qualidade e a confiabilidade dos resultados obtidos, na qual a torna uma ferramenta valiosa para a solução proposta no trabalho.

% ---
\section{Da União da Internet com o Setor Industrial}
% ---
% Avanço da Internet e União da Internet + Setor Industrial
% ========================================================
% QUESTÃO: 

O panorama atual de avanço da internet e a convergência entre a internet e o setor industrial representam um arco significativo na era da Engenharia de Computação [REFERÊNCIA]. Este fenômeno transformador tem sido impulsionado pela fusão das tecnologias da informação com os processos industriais, dando origem a conceitos como Indústria 4.0. 

No âmbito desta ferramenta é aplicado a intersecção dessas duas esferas, onde os conceitos de práticas industriais, reconciliação, análise e qualidade de dados se integram à internet na qual é extraído deles o seu maior forte, como uma maior integralidade com outros sistemas por meio de APIs (interfaces de programação de aplicativos), melhor interatividade entre os elementos do sistema, promovendo uma comunicação mais dinâmica e eficaz, aumento da eficiência operacional e facilidade na gestão de processos [REFERÊNCIA]. Essa sinergia possibilita a criação de ecossistemas industriais mais conectados nos quais os dados relevantes podem ser tratados de forma segura e eficiente. [REFERÊNCIA].

% ---
\section{Do Panorama Atual e o Futuro}
% ---
% O panorama atual sugere que as industrias vão crescer no país.
% ========================================================
% QUESTÃO: 

O horizonte atual, delineado pelos recentes avanços tecnológicos e inovações sustenta a pespectiva otimista que as indústrias estão destinadas a experimentar um crescimento substancial no país. A convergência entre a internet e o setor industrial, representa um catalisador significativo para a modernização e eficiência operacional. A integração de práticas avançadas de desenvolvimento de soluções voltadas a usabilidade e ambiente de desenvolvimento com controle computacional, como a reconciliação de dados e análise preditiva, impulsiona a qualidade e consistência dos processos produtivos.

Além disso, a aplicação da internet nas práticas industriais não só fortalece a competitividade das empresas mas também desempenha um papel crucial na expansão econômica do país. A capacidade de adotar tecnologias inovadoras como a automação avançada, coloca as indústrias em uma posição estratégica para atender às crescentes demandas do mercado e elevar a produtividade [REFERÊNCIA]. Nesse sentido é plausível afirmar que diante do atual cenário tecnológico e das tendências emergentes, é indubitável a necessidade e importância da ferramenta desenvolvida nesse trabalho.



\chapter[Desenvolvimento]{Desenvolvimento}

Vestibulum mollis pulvinar venenatis. Morbi maximus interdum ipsum ac tincidunt. Cras vulputate volutpat ex, eu consequat tellus scelerisque a. Integer a commodo purus, eleifend suscipit nibh. Nulla vitae varius mauris. In hac habitasse platea dictumst. Nullam euismod et nulla nec aliquet. Nam placerat faucibus quam ac vehicula. Etiam vel ex massa. Donec faucibus, dolor sed eleifend gravida, nisl urna facilisis dui, et volutpat turpis nisi eu augue. Aliquam egestas purus at nulla bibendum efficitur. Nam mollis vitae arcu a interdum. In vel gravida nunc. Nam eu risus ut odio suscipit varius. Nunc vehicula, augue tempor congue accumsan, metus felis commodo dolor, a pharetra risus elit in erat.

Sed eros urna, accumsan eget malesuada at, vestibulum at turpis. Nullam hendrerit justo orci, et elementum neque pulvinar sed. Mauris odio nunc, fermentum scelerisque elit vitae, ornare varius turpis. Integer id aliquam urna. Mauris diam arcu, cursus sit amet augue id, pretium mollis risus. Nulla facilisi. Suspendisse id erat velit. Mauris suscipit tempor metus, vel venenatis velit malesuada ac. Donec vel eros libero.

\chapter[Resultados]{Resultados}

Com o advento de ferramentas construídas inteiramente na base web, surge um espaço p


- Existe um acréscimo na construção de industrias.


- Panorama atual sugere que está acontecendo algo X. Esse comportamento X influencia diretamente Y. Neste sentido, é importante trabalhar em Y.


% ---
\section{Caso estudado}
% ---


% ---
\section{Justificativas}
% ---


% ---
\section{Objetivos}
% ---

-> Falar o que é uma Reconciliação de Dados.

-> Falar o método matemático utilizado.

-> Falar que vamos atacar esse problemas de duas frentes, a frente matemática e a frente computacional.

Este documento e seu código-fonte são exemplos de referência de uso da classe \textsf{faeng.cls} (baseada na classe \textsf{eesc.cls}) e do pacote \textsf{abntex2cite}. O documento exemplifica a elaboração de trabalho acadêmico produzido conforme a \ac{ABNT} \ac{NBR} 14724:2011 \emph{Informação e documentação - Trabalhos acadêmicos - Apresentação}.

Este modelo é uma implementação das normas para produção de textos estabelecida pela Faculdade de Engenharia da Universidade Federal de Mato Grosso, Campus Universitário de Várzea Grande.

\chapter[Conclusão]{Conclusão}

Vestibulum mollis pulvinar venenatis. Morbi maximus interdum ipsum ac tincidunt. Cras vulputate volutpat ex, eu consequat tellus scelerisque a. Integer a commodo purus, eleifend suscipit nibh. Nulla vitae varius mauris. In hac habitasse platea dictumst. Nullam euismod et nulla nec aliquet. Nam placerat faucibus quam ac vehicula. Etiam vel ex massa. Donec faucibus, dolor sed eleifend gravida, nisl urna facilisis dui, et volutpat turpis nisi eu augue. Aliquam egestas purus at nulla bibendum efficitur. Nam mollis vitae arcu a interdum. In vel gravida nunc. Nam eu risus ut odio suscipit varius. Nunc vehicula, augue tempor congue accumsan, metus felis commodo dolor, a pharetra risus elit in erat.

Sed eros urna, accumsan eget malesuada at, vestibulum at turpis. Nullam hendrerit justo orci, et elementum neque pulvinar sed. Mauris odio nunc, fermentum scelerisque elit vitae, ornare varius turpis. Integer id aliquam urna. Mauris diam arcu, cursus sit amet augue id, pretium mollis risus. Nulla facilisi. Suspendisse id erat velit. Mauris suscipit tempor metus, vel venenatis velit malesuada ac. Donec vel eros libero.

%% ==============================
%% ELEMENTOS PÓS-TEXTUAIS
%% ------------------------------
\postextual
%% Referências bibliográficas
%% ------------------------------
\bibliography{abntex2-modelo-references}

%% Glossário
%% ------------------------------
%\glossary
%% ==============================

%% Apêndices
\begin{apendicesenv}
\partapendices %% Imprime uma página indicando o início dos apêndices
\chapter{Quisque libero justo} %% Divisão em capítulos, como no restante
\lipsum[1-5]
\end{apendicesenv}
%% ==============================

%% ==============================
%% Anexos
%% ------------------------------
%% Inicia os anexos
%% ------------------------------
\begin{anexosenv}
\partanexos  %% Imprime uma página indicando o início dos anexos
\chapter{Morbi ultrices rutrum lorem.} %% Divisão em capítulos.
\lipsum[1-25]
\section{Test} %% Divisão em sessões.
\lipsum[1-20]
\end{anexosenv}
%% ==============================

\end{document}