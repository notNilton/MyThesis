%% Este trabalho é uma adequação das normas de TCC
%% da Universidade Federal de Mato Grosso (Faculdade de Engenharia) 
%% de acordo com a Norma ABNT.

%% ========================================================================
%% Opções: \documentclass[tipo/curso]{faeng}
%% ------------------------------------------------------------------------
%% Tipo:
%% 	tcc:         Formata documento para TCC
%%	qualitcc:    Formata documento para qualificação de TCC
%% ------------------------------------------------------------------------
%% Curso:
%%   eq:     Engenharia Química
%%   em:     Engenharia de Minas
%%   ec:     Engenharia de Computação
%%   et:     Engenharia de Transportes
%%   eca:    Engenharia de Controle e Automação
%% ------------------------------------------------------------------------
\documentclass[tcc/eca]{faeng}
%% ========================================================================

%% ========================================================================
%% PACOTES
%% ------------------------------------------------------------------------
%% Pacotes fundamentais 
%% ------------------------------------------------------------------------
\usepackage{cmap}			% Mapear caracteres especiais no PDF
\usepackage{lmodern}		% Usa a fonte Latin Modern			
\usepackage{makeidx}        % Cria o indice
\usepackage{hyperref}  		% Controla a formação do índice
\usepackage{lastpage}		% Usado pela Ficha catalográfica
\usepackage{indentfirst}	% Indenta o primeiro parágrafo de cada seção.
\usepackage{nomencl} 		% Lista de simbolos
\usepackage{graphicx}		% Inclusão de gráficos
\usepackage[brazil]{babel}  % Codificação e uso de caracteres especiais.
\usepackage[utf8]{inputenc}
%% ------------------------------------------------------------------------
%% Pacotes adicionais, usados apenas no âmbito do Modelo faeng
%% ------------------------------------------------------------------------
\usepackage{lipsum}				       % para geração de dummy text
\usepackage[printonlyused]{acronym}
\usepackage{xcolor}
\usepackage{booktabs}
\usepackage{multirow}
%% ========================================================================

%% ========================================================================
%% Informações de dados para CAPA e FOLHA DE ROSTO
%% ------------------------------------------------------------------------
%% Título:
%%	1. Título em português
%%	2. Título em inglês
\titulo{Aplicação web de dashboard de reconciliação de dados utilizando o método de minimização de funções multivariáveis com multiplicadores de Lagrange}{Data reconciliation dashboard web application using the multivariable function minimization method with Lagrange multipliers}
%% ------------------------------------------------------------------------
%% Autor:
%%	1. Nome completo do autor
%%	2. Formato de nome para bibliografia
\autor{Nilton Aguiar dos Santos}{Santos, Nilton}
%% ------------------------------------------------------------------------
%% Cidade
\local{Várzea Grande}
%% ------------------------------------------------------------------------
%% Ano de defesa
\data{2024}
%% ------------------------------------------------------------------------
%% Nome do orientador
\orientador{João Gustavo Coelho Pena}
%% Nome do coorientador
%\coorientador{Nome completo do coorientador}
%% ========================================================================

%% ========================================================================
%% compila o indice
%% ------------------------------------------------------------------------
\makeindex
%% ========================================================================

%% ========================================================================
%% Compila a lista de abreviaturas e siglas
%% ------------------------------------------------------------------------
\makenomenclature
%% ========================================================================

%% ========================================================================
%% Inserir ficha catalográfica
%%
%% Caso o comando \inserirfichacatalografica seja definido, a ficha catalográfica
%% será inserida atrás da folha de rosto. Caso contrário a página será deixada em branco.
%%
%% CUIDADO: Esta opção deve ser preenchida antes do comando \maketitle
%% ------------------------------------------------------------------------
%\inserirfichacatalografica{fichaCatalografica.pdf}
%% ========================================================================

%% ========================================================================
%% Inserir folha de aprovação
%%
%% Caso o comando \inserirfolhaaprovacao seja definido, a a folha de aprovação
%% será inserida.
%% CUIDADO: Esta opção deve ser preenchida antes do comando \maketitle
%% ------------------------------------------------------------------------
%\inserirfolhaaprovacao{folhaAprovacao.pdf}
%% ========================================================================

%% ========================================================================
%% INÍCIO DO DOCUMENTO
%% ------------------------------------------------------------------------
\begin{document}
%% ------------------------------------------------------------------------
%% ELEMENTOS PRÉ-TEXTUAIS
%% ------------------------------------------------------------------------
\pretextual
%% ------------------------------------------------------------------------
%% Insere Capa, Folha de rosto, Ficha catalográfica (se inserida)
%% e folha de aprovação (se inserida).
%% ------------------------------------------------------------------------
\maketitle
%% ------------------------------------------------------------------------
%% Dedicatória
%% ------------------------------------------------------------------------
\imprimirdedicatoria{Dedico a mim mesmo, pq eu sou brabo, brabadíssimo, o brabo dos brabos.}
%% ------------------------------------------------------------------------
%% Agradecimentos
%% ------------------------------------------------------------------------
\imprimiragradecimentos{
Agradeço ao Programa de Pós-Graduação em Engenharia
Elétrica da Universidade Federal do Espírito Santo pela
oportunidade de aprimoramento de minha formação profis-
sional, em especial, por favorecer a realização de um sonho.
Agradeço à CAPES (Coordenação de Aperfeiçoamento
Pessoal de Nível Superior), pela bolsa concedida, sem a
qual não teria sido possível minha dedicação ao presente
trabalho.

Agradeço à ArcelorMittal pela oportunidade de poder rea-
lizar esta pesquisa. Certamente, o sucesso desta pesquisa
seria incerto caso não fosse possível ter acesso às depen-
dências da empresa e ao seu corpo técnico.

Agradeço aos professores e funcionários do Departamento de Engenharia Elétrica, principalmente pela amizade e atenção.

Agradeço ao professor Helder Roberto de Oliveira Rocha pela confiança, pela atenção e prontidão no auxílio para êxito deste projeto.

Sou grato, especialmente, ao professor José Leandro Félix Salles pelo zelo, dedicação e parceria no desenvolvimento desta pesquisa e realização desta tese. Essa conquista foi possível por conta do seu entusiasmo, seu cuidado e esforço, pela paciência, pelo apoio técnico e psicológico. Aprendi a admirá-lo sob diversas perspectivas, e sou grato por tudo isso.

Sou grato também aos engenheiros Valter Barbosa de Oliveira Júnior e Murilo Siqueira Muniz Teixeira da Silva, que deram amplo suporte à modelagem de processos, análises, simulações e testes dos modelos de previsão e otimização. Sou grato à minha esposa, aos meus pais e irmãs pelo carinho, pelo auxílio e pelo entusiasmo.

Sou grato ao SENHOR Adonai Eloheinu, O Santo Deus de Avraham, e Yitz’chak, e Yisra’el. Sou grato por Suas bençãos, pela graça e favorecimento em minha vida.
}
%% ------------------------------------------------------------------------
%% Epígrafe
%% ------------------------------------------------------------------------
\imprimirepigrafe{
		``Frase.''\\
		(Autor)
}
%% ========================================================================

%% ========================================================================
%% RESUMO e ABSTRACT
%% ------------------------------------------------------------------------
%% Resumo em português
%% ------------------------------------------------------------------------ 
\begin{resumo}{Reconciliação de Dados. Análise de Qualidade de Dados. Desenvolvimento Web. Multiplicadores de Lagrange.}

Neste trabalho de conclusão de curso, desenvolveu-se um software Dashboard de análise de qualidade de dados e reconciliação de dados utilizando as técnicas de minimização de funções multivariáveis pelo método dos multiplicadores de Lagrange. A solução é desenvolvida em duas principais frentes, os conceitos computacionais modernos de experiência de usuário, agilidade no comportamento e (um monte mais de coisas), e a segunda frente sendo da aplicação dos cálculos matemáticos e estatísticos voltados ao problema de reconciliação de dados e qualidade de dados. Por meio do Dashboard de Reconciliação de Dados é possível modelar todo um processo industrial e a partir disso tanto calcular a reconciliação de dados como analisar a qualidade de dados em tempo real. O software é uma solução inovadora dado que não há um direto competidor para as funções oferecidas em um ambiente mais acessível e ágil. Ao longo deste trabalho o processo filosófico do desenvolvimento, os cálculos matemáticos, os conceitos estatísticos e computacionais e a lógica do software aplicado como solução são explicados de forma detalhada. Exemplos do código funcional e considerações finais sobre o trabalho realizado são apresentados nos últimos capítulos.

\end{resumo}
%% ------------------------------------------------------------------------
%% Resumo em inglês
%% ------------------------------------------------------------------------
\begin{abstract}{Data Reconciliation. Data Quality Analysis. Web Development. Lagrange Multipliers.}

Abstract.
	
\end{abstract}
%% ========================================================================

%% ========================================================================
%% inserir lista de ilustrações
%% ------------------------------------------------------------------------
\listailustracoes
%% ========================================================================

%% ========================================================================
%% inserir lista de tabelas
%% ------------------------------------------------------------------------
\listatabelas
%% ========================================================================

%% ========================================================================
%% inserir lista de abreviaturas e siglas
%% ------------------------------------------------------------------------
\listasiglas{abreviaturas}
%% ========================================================================

%% ========================================================================
%% inserir o sumario
%% ------------------------------------------------------------------------
\sumario
%% ========================================================================

%% ========================================================================
%% ELEMENTOS TEXTUAIS
%% ------------------------------------------------------------------------
\mainmatter
%% ========================================================================

%% ========================================================================
%% Capitulos 
%% ------------------------------------------------------------------------
%% Capítulo externo
\include{introducao}
%% ------------------------------------------------------------------------
%% Capítulo
\chapter[Capítulo]{Capítulo}
Esse também é um capítulo.
%% ------------------------------------------------------------------------
%% Capítulo
\include{abntex2-modelo-include-comandos}
%% ------------------------------------------------------------------------
%% Conclusão
\chapter[Conclusão]{Conclusão}
A conclusão também é um capítulo.
%% ========================================================================

%% ========================================================================
%% ELEMENTOS PÓS-TEXTUAIS
%% ------------------------------------------------------------------------
\postextual
%% ========================================================================

%% ========================================================================
%% Referências bibliográficas
%% ------------------------------------------------------------------------
\bibliography{abntex2-modelo-references}
%% ========================================================================

%% ========================================================================
%% Glossário
%% ------------------------------------------------------------------------
%\glossary
%% ========================================================================

%% ========================================================================
%% Apêndices
%% ------------------------------------------------------------------------
%% Inicia os apêndices
%% ------------------------------------------------------------------------
\begin{apendicesenv}
\partapendices %% Imprime uma página indicando o início dos apêndices
\chapter{Quisque libero justo} %% Divisão em capítulos, como no restante
\lipsum[1-5]
\end{apendicesenv}
%% ========================================================================

%% ========================================================================
%% Anexos
%% ------------------------------------------------------------------------
%% Inicia os anexos
%% ------------------------------------------------------------------------
\begin{anexosenv}
\partanexos  %% Imprime uma página indicando o início dos anexos
\chapter{Morbi ultrices rutrum lorem.} %% Divisão em capítulos.
\lipsum[1-25]
\section{Test} %% Divisão em sessões.
\lipsum[1-20]
\end{anexosenv}
%% ========================================================================

\end{document}