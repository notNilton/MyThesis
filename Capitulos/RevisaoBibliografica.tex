\chapter[Revisão Bibliográfica]{Revisão Bibliográfica}

\section{Reconciliação de Dados no âmbito industrial}
% Reconciliação de Dados
% ========================================================
% QUESTÃO:

Já no começo da década de 60 se entendeu a importância do controle de processos químicos industriais por computadores utilizando técnicas matemáticas (KUEHN, 1961), dessa forma surge a área da computação voltada à reconciliação de dados, na qual há a comparação, validação e correção de informações coletadas em diferentes pontos do processo, afim de determinar a consistência dos dados, a qualidade dos mesmos ou até otimizar os processos (NARASIMHAN, 2000).

Ao longo das últimas décadas, os métodos de reconciliação de dados evoluíram significativamente, acompanhando os avanços tecnológicos e as demandas crescentes da indústria [REFERÊNCIA]. Com o advento de sistemas de automação mais avançados, sensores inteligentes e a proliferação de dispositivos conectados, a quantidade de dados gerados nas operações industriais aumentou drasticamente [REFERÊNCIA]. Nesse contexto, a reconciliação, análise e gestão de dados tornaram-se ferramentas indispensáveis para garantir a integridade e a confiabilidade das informações coletadas em tempo real[REFERÊNCIA].

Na era contemporânea, a reconciliação de dados desempenha um papel crucia na otimização dos processos industriais, contribuindo para a eficiência operacional e a redução e custos [REFERÊNCIA]. Sistemas avançados de reconciliação não apenas comparam e validam dados mas também utilizam algoritmos sofisticados para identificar padrões, tendências e anomalias [REFERÊNCIA]. Essa capacidade analítica permite que as indústrias ajam proativamente, antecipando-se a problemas potenciais, otimizando a produção e melhorando a qualidade dos produtos finais [REFERÊNCIA].

\section{Utilização do método dos multiplicadores de Lagrange}
% ---
% Reconciliação de dados utilizando o método dos multiplicadores de Lagrange
% ========================================================
% QUESTÃO: Será que é necessário uma maior explicação sobre esse tópico?

No sistema em questão a reconciliação de dados vai ser feita com a minimização de funções multivariáveis utilizando método dos multiplicadores de Lagrange, desenvolvida pelo matemático Joseph Louis Lagrange (1739-1813), na qual desenvolveu um método de encontrar o mínimo ou máximo de uma função multivariável sujeita a uma ou várias condições de restrição.

Nesse contexto, a aplicação do método dos multiplicadores de Lagrange destaca-se como uma abordagem matemática sofisticada e eficaz para resolver problemas complexos de reconciliação de dados. A técnica proporciona uma estrutura robusta para lidar com situações em que é necessário otimizar uma função multivariável sujeita a restrições específicas. Ao utilizar os multiplicadores de Lagrange, o sistema ganha a capacidade de encontrar soluções que atendam simultaneamente às condições impostas, proporcionando uma reconciliação precisa e eficiente dos dados envolvidos. Essa metodologia, fundamentada em princípios matemáticos sólidos, contribui para aprimorar a qualidade e a confiabilidade dos resultados obtidos, na qual a torna uma ferramenta valiosa para a solução proposta no trabalho.

\section{Desenvolvimento de uma ferramenta online}
% Desenvolvimento de uma ferramenta online
% ========================================================
% QUESTÃO: 

Vestibulum mollis pulvinar venenatis. Morbi maximus interdum ipsum ac tincidunt. Cras vulputate volutpat ex, eu consequat tellus scelerisque a. Integer a commodo purus, eleifend suscipit nibh. Nulla vitae varius mauris. In hac habitasse platea dictumst. Nullam euismod et nulla nec aliquet. Nam placerat faucibus quam ac vehicula. Etiam vel ex massa. Donec faucibus, dolor sed eleifend gravida, nisl urna facilisis dui, et volutpat turpis nisi eu augue. Aliquam egestas purus at nulla bibendum efficitur. Nam mollis vitae arcu a interdum. In vel gravida nunc. Nam eu risus ut odio suscipit varius. Nunc vehicula, augue tempor congue accumsan, metus felis commodo dolor, a pharetra risus elit in erat.

Sed eros urna, accumsan eget malesuada at, vestibulum at turpis. Nullam hendrerit justo orci, et elementum neque pulvinar sed. Mauris odio nunc, fermentum scelerisque elit vitae, ornare varius turpis. Integer id aliquam urna. Mauris diam arcu, cursus sit amet augue id, pretium mollis risus. Nulla facilisi. Suspendisse id erat velit. Mauris suscipit tempor metus, vel venenatis velit malesuada ac. Donec vel eros libero.

\section{A sinergia entre a indústria e a internet}
% A sinergia entre a indústria e a internet
% ========================================================
% QUESTÃO: 

O panorama atual de avanço da internet e a convergência entre a internet e o setor industrial representam um arco significativo na era da Engenharia de Computação [REFERÊNCIA]. Este fenômeno transformador tem sido impulsionado pela fusão das tecnologias da informação com os processos industriais, dando origem a conceitos como Indústria 4.0. 

No âmbito desta ferramenta é aplicado a intersecção dessas duas esferas, onde os conceitos de práticas industriais, reconciliação, análise e qualidade de dados se integram à internet na qual é extraído deles o seu maior forte, como uma maior integralidade com outros sistemas por meio de APIs (interfaces de programação de aplicativos), melhor interatividade entre os elementos do sistema, promovendo uma comunicação mais dinâmica e eficaz, aumento da eficiência operacional e facilidade na gestão de processos [REFERÊNCIA]. Essa sinergia possibilita a criação de ecossistemas industriais mais conectados nos quais os dados relevantes podem ser tratados de forma segura e eficiente. [REFERÊNCIA].

O horizonte atual, delineado pelos recentes avanços tecnológicos e inovações sustenta a perspectiva otimista que as indústrias estão destinadas a experimentar um crescimento substancial no país. A convergência entre a internet e o setor industrial, representa um catalisador significativo para a modernização e eficiência operacional. A integração de práticas avançadas de desenvolvimento de soluções voltadas a usabilidade e ambiente de desenvolvimento com controle computacional, como a reconciliação de dados e análise preditiva, impulsiona a qualidade e consistência dos processos produtivos.

Além disso, a aplicação da internet nas práticas industriais não só fortalece a competitividade das empresas mas também desempenha um papel crucial na expansão econômica do país. A capacidade de adotar tecnologias inovadoras como a automação avançada, coloca as indústrias em uma posição estratégica para atender às crescentes demandas do mercado e elevar a produtividade [REFERÊNCIA]. Nesse sentido é plausível afirmar que diante do atual cenário tecnológico e das tendências emergentes, é indubitável a necessidade e importância da ferramenta desenvolvida nesse trabalho.