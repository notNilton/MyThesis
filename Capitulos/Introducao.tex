\chapter[Introdução]{Introdução}
% Introdução
% ========================================================
% QUESTÃO:
O panorama atual sugere que o setor industrial brasileiro continua a crescer (Tomás Torezani, 2020) porém com ele deve seguir paralelamente um avanço tecnológico para que ela consiga aumentar o nível de produtividade dessas indústrias (Marcos Lisboa, 2021), dessa forma se origina o funcionalidade do trabalho de conclusão de curso DRD (Dashboard de Reconciliação de Dados), na qual ele unifica a área de análise de qualidade dos dados e a reconciliação desses mesmos com a tecnologia oriunda dos avanços tecnológicos na área de desenvolvimento web.

\section{Objetivos Gerais}
% Objetivos Gerais
% ========================================================
% QUESTÃO:

Considerado a necessidade de inovação na área industrial e da união entre essas duas áreas, o objetivo deste trabalho é apresentar o desenvolvimento de um sistema online de reconciliação de dados utilizando métodos matemáticos criados por Lagrange.

\subsubsection{Objetivos específicos}
% Objetivos específicos
% ========================================================
% QUESTÃO:

Para alcançar o objetivo proposto neste trabalho, cuja finalidade é desenvolver uma ferramenta online de reconciliação de dados é necessário especificar os seguintes objetivos específicos:

a) Realizar a pesquisa de estado da arte da área de desenvolvimento de software para tal função.

b) Analisar e compreender as metodologias e algoritmos existentes na área de reconciliação de dados, destacando suas aplicações, vantagens e limitações. A pesquisa de estado da arte é crucial para identificar as tendências mais recentes, as melhores práticas e as lacunas existentes, proporcionando uma base sólida para o desenvolvimento da ferramenta online.

c) Selecionar e adaptar as tecnologias adequadas para a implementação da ferramenta online de reconciliação de dados. Isso envolverá a avaliação de linguagens de programação, frameworks e plataformas que melhor se adéquem aos requisitos específicos da aplicação, considerando aspectos como eficiência, escalabilidade e segurança.

d) Projetar a arquitetura da ferramenta, delineando os componentes principais, a interação entre eles e a lógica de funcionamento. A clareza na definição da arquitetura será essencial para garantir a eficácia operacional da ferramenta, bem como para facilitar futuras atualizações e expansões.

e) Desenvolver a ferramenta online de reconciliação de dados, implementando os algoritmos e funcionalidades identificados na pesquisa de estado da arte. Durante essa fase, é crucial garantir a usabilidade, a integridade dos dados e a eficiência do sistema, atendendo aos padrões de qualidade e performance esperados.

f) Realizar testes abrangentes para validar a eficácia e confiabilidade da ferramenta. Isso incluirá testes de integração, testes de segurança e simulações de casos práticos para verificar a capacidade da ferramenta em lidar com diferentes cenários e volumes de dados.

g) Elaborar uma documentação completa, abrangendo desde o processo de desenvolvimento até as instruções de uso da ferramenta. Uma documentação robusta e clara será essencial para facilitar a compreensão, manutenção e futuras implementações relacionadas à ferramenta de reconciliação de dados.

\section{Justificativa}
% Justificativa
% ========================================================
% QUESTÃO:

A ferramenta online de reconciliação de dados baseia-se na interseção entre a indústria e tecnologia da internet 4.0. Observa-se a necessidade premente de ferramentas especializadas que possam acompanhar e potencializar essa convergência, contribuindo para a eficiência operacional, qualidade dos processos industriais e aprimoramento da competitividade das empresas. A atual revolução industrial demanda soluções tecnológicas avançadas que permitam a integração e análise eficiente de dados em tempo real. A solução DRD torna-se crucial para manter a integridade das informações, garantindo a consistência e confiabilidade necessárias para operações industriais avançadas.

A utilização de uma ferramenta online de reconciliação de dados abre caminho para a otimização de processos, unificação com outras ferramentas, redução de erros, monitoramento em tempo real e, consequentemente, aumento da eficiência operacional. Isso não apenas reduz custos operacionais, mas também posiciona as empresas em um patamar competitivo mais vantajoso. E a integração dessa ferramenta com outras permite não apenas a reconciliação de dados, mas também a extração de insights valiosos para a tomada de decisões estratégicas. Essa sinergia contribui para a inovação contínua e a adaptação rápida às mudanças no ambiente industrial.

O desenvolvimento de uma ferramenta posiciona-se em consonância com as tendências globais de transformação digital e automação industrial. Isso não apenas fortalece a competitividade das empresas no mercado nacional, mas também as prepara para atuar de maneira efetiva em uma escala internacional. Em suma, diante do cenário tecnológico atual e das demandas evolutivas da indústria, a criação de uma ferramenta online de reconciliação de dados emerge como uma resposta estratégica para potencializar a interseção entre a Internet e o setor industrial, impulsionando a eficiência e a inovação nas práticas industriais contemporâneas.