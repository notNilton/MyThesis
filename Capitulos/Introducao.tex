\chapter[Introdução]{Introdução}

[Existem muitas indústrias altamente tecnologicas sendo feitas no estado de Mato Grosso, com isso surge um vácuo em algumas áreas ainda não inteiramente desenvolvidas]Com o advento de ferramentas construídas inteiramente na base web, surge um espaço

A 
-> Contar a História de Reconciliação de Dados

-> Contar a História Laplace

-> Contar a História de Aplicação Web

- Existe um acréscimo na construção de industrias.


- Panorama atual sugere que está acontecendo algo X. Esse comportamento X influencia diretamente Y. Neste sentido, é importante trabalhar em Y.


% ---
\section{Caso estudado}
% ---


% ---
\section{Justificativas}
% ---


% ---
\section{Objetivos}
% ---

-> Falar o que é uma Reconciliação de Dados.

-> Falar o método matemático utilizado.

-> Falar que vamos atacar esse problemas de duas frentes, a frente matemática e a frente computacional.

Este documento e seu código-fonte são exemplos de referência de uso da classe \textsf{faeng.cls} (baseada na classe \textsf{eesc.cls}) e do pacote \textsf{abntex2cite}. O documento exemplifica a elaboração de trabalho acadêmico produzido conforme a \ac{ABNT} \ac{NBR} 14724:2011 \emph{Informação e documentação - Trabalhos acadêmicos - Apresentação}.

Este modelo é uma implementação das normas para produção de textos estabelecida pela Faculdade de Engenharia da Universidade Federal de Mato Grosso, Campus Universitário de Várzea Grande.