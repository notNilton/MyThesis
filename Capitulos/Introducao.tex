\chapter[Introdução]{Introdução}

% ---
\section{Visão Geral | Um problema de três vanguardas}
% ---

Já no começo da década de 60 se entendeu a importância do controle de processos químicos industriais por computadores utilizando técnicas matemáticas (KUEHN, 1961), dessa forma surge a área da computação voltada à reconciliação de dados, na qual há a comparação, validação e correção de informações coletadas em diferentes pontos do processo, afim de determinar a consistência dos dados, a qualidade dos mesmos ou até otimizar os processos (NARASIMHAN, 2000).

Ao longo das últimas décadas, os métodos de reconciliação de dados evoluíram significativamente, acompanhando os avanços tecnológicos e as demandas crescentes da indústria [REFERÊNCIA]. Com o advento de sistemas de automação mais avançados, sensores inteligentes e a proliferação de dispositivos conectados, a quantidade de dados gerados nas operações industriais aumentou drasticamente [REFERÊNCIA]. Nesse contexto, a reconciliação, análise e gestão de dados tornaram-se ferramentas indispensáveis para garantir a integridade e a confiabilidade das informações coletadas em tempo real[REFERÊNCIA].

Na era contemporânea, a reconciliação de dados desempenha um papel crucia na otimização dos processos industriais, contribuindo para a eficiência operacional e a redução e custos [REFERÊNCIA]. Sistemas avançados de reconciliação não apenas comparam e validam dados mas também utilizam algoritmos sofisticados para identificar padrões, tendências e anomalias [REFERÊNCIA]. Essa capacidade analítica permite que as indústrias ajam proativamente, antecipando-se a problemas potenciais, otimizando a produção e melhorando a qualidade dos produtos finais [REFERÊNCIA].

Da mesma forma é possível observar o crescimento do setor da Indústria 4.0 q

-> O panorama atual sugere que as industrias vão crescer no país, ou seja.

O panorama atual sugere que o setor químico industrial está passando por um crescimento nos últimos anos 
mas que é necessário um avanço tecnológico par, o que está faltando. [REFERÊNCIA]

-> Agora vem a parte da área tecnológica importante, 

-> Precisa-se investir nessa área, como diz tal artigo.




da utilização de técnicas computacionais em um processo industrial afim de ter o controle em relação a qualidade e análise dos dados de forma automatizada e metódica. Surge juntamente com isso a idéia de utilizar conceitos de reconciliação de dados, uma técnica, como também da análise do mesmo (KUEHN, 1960). [AUMENTAR A QUANTIDADE DE BACKGROUND]
Embora seja antiga essa 

[Existem muitas indústrias altamente tecnologicas sendo feitas no estado de Mato Grosso, com isso surge um vácuo em algumas áreas ainda não inteiramente desenvolvidas]
Com o advento de ferramentas construídas inteiramente na base web, surge um espaço

A 
-> Contar a História de Análise e Reconciliação de Dados

-> Contar a História Laplace

-> Contar a História de Aplicação Web

- Existe um acréscimo na construção de industrias.


- Panorama atual sugere que está acontecendo algo X. Esse comportamento X influencia diretamente Y. Neste sentido, é importante trabalhar em Y.


% ---
\section{Justificativas}
% ---


% ---
\section{Objetivos}
% ---

-> Falar o que é uma Reconciliação de Dados.

-> Falar o método matemático utilizado.

-> Falar que vamos atacar esse problemas de duas frentes, a frente matemática e a frente computacional.

Este documento e seu código-fonte são exemplos de referência de uso da classe \textsf{faeng.cls} (baseada na classe \textsf{eesc.cls}) e do pacote \textsf{abntex2cite}. O documento exemplifica a elaboração de trabalho acadêmico produzido conforme a \ac{ABNT} \ac{NBR} 14724:2011 \emph{Informação e documentação - Trabalhos acadêmicos - Apresentação}.

Este modelo é uma implementação das normas para produção de textos estabelecida pela Faculdade de Engenharia da Universidade Federal de Mato Grosso, Campus Universitário de Várzea Grande.