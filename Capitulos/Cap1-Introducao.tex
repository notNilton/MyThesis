\chapter{Introdução} \label{Introducao}

O panorama atual sugere que o setor industrial brasileiro é um dos mais importantes para a economia nacional, sendo responsável por cerca de 20\% do PIB e 70\% das exportações de acordo com o IBGE (Instituto Brasileiro de Geografia e Estatística). Segundo o economista Tomás Torezani, o panorama atual sugere que esse setor continua a crescer, apesar dos desafios impostos pela pandemia de Covid-19 e pela crise fiscal. No entanto, para manter a competitividade e o desenvolvimento sustentável, é preciso investir em inovação e tecnologia.

Nesse sentido, o trabalho de conclusão de curso DRD (Dashboard de Reconciliação de Dados) propõe uma solução para melhorar a qualidade e a confiabilidade dos dados utilizados pelas indústrias. O software é uma ferramenta web que permite a validação, a limpeza e a reconciliação dos dados provenientes de diferentes fontes e sistemas. Assim, ele facilita a análise e a tomada de decisão baseada em informações precisas e atualizadas.

O DRD se insere no contexto da transformação digital que vem ocorrendo nas indústrias brasileiras, buscando aumentar o nível de produtividade e eficiência. De acordo com o presidente do Insper, Marcos Lisboa, o avanço tecnológico é fundamental para que as indústrias possam se adaptar às mudanças do mercado e às demandas dos consumidores. Por isso, a solução representa uma contribuição relevante para o aprimoramento do setor industrial brasileiro, aliando tecnologia e qualidade dos dados.

\section{Objetivos}

\subsection{\textit{Objetivo geral}}

Desenvolver e implementar um software web para a validação, a limpeza e a reconciliação dos dados utilizados pelas indústrias brasileiras, usando métodos matemáticos, estatísticos e físicos, visando aumentar a qualidade e a confiabilidade das informações para a análise e a tomada de decisão.

\subsection{\textit{Objetivos específicos}}

Para alcançar o objetivo proposto neste trabalho, cuja finalidade é desenvolver uma ferramenta online de reconciliação de dados é necessário seguir os seguintes objetivos específicos:

\begin{itemize}

\item Realizar a pesquisa de estado da arte da área de desenvolvimento sobre as principais técnicas e ferramentas de validação, limpeza e reconciliação de dados. 

\item Analisar e compreender as metodologias e algoritmos existentes na área de reconciliação de dados, destacando suas aplicações, vantagens e limitações. A pesquisa de estado da arte é crucial para identificar as tendências mais recentes, as melhores práticas e as lacunas existentes, proporcionando uma base sólida para o desenvolvimento da ferramenta online.

\item Selecionar e adaptar as tecnologias adequadas para a implementação da ferramenta online de reconciliação de dados. Isso envolverá a avaliação de linguagens de programação, frameworks e plataformas que melhor se adéquem aos requisitos específicos da aplicação, considerando aspectos como eficiência, escalabilidade e segurança.

\item Projetar a arquitetura da ferramenta, delineando os componentes principais, a interação entre eles e a lógica de funcionamento. A clareza na definição da arquitetura será essencial para garantir a eficácia operacional da ferramenta, bem como para facilitar futuras atualizações e expansões.

\item Desenvolver a ferramenta online de reconciliação de dados, implementando os algoritmos e funcionalidades identificados na pesquisa de estado da arte. Durante essa fase, é crucial garantir a usabilidade, a integridade dos dados e a eficiência do sistema, atendendo aos padrões de qualidade e performance esperados.

\item Realizar testes abrangentes para validar a eficácia e confiabilidade da ferramenta. Isso incluirá testes de integração, testes de segurança e simulações de casos práticos para verificar a capacidade da ferramenta em lidar com diferentes cenários e volumes de dados.

\item Elaborar uma documentação completa, abrangendo desde o processo de desenvolvimento até as instruções de uso da ferramenta. Uma documentação robusta e clara será essencial para facilitar a compreensão, manutenção e futuras implementações relacionadas à ferramenta de reconciliação de dados.
\end{itemize}

\section{Estado da Arte}

Em meio ao processo de elaboração do trabalho de conclusão de curso, um aspecto fundamental que merece destaque é a investigação das tendências tecnológicas emergentes nas áreas relevantes para o estudo. Essa pesquisa não apenas fornece uma visão abrangente do cenário atual, mas também ajuda a identificar oportunidades para inovação e melhoria.

As tendências tecnológicas são indicadores poderosos do progresso em qualquer campo de estudo e podem ser representadas tanto na área comercial direta, como na Indústria Química, nas áreas adjacentes, como em conceitos de contabilidade além de pesquisas cientificas. Elas refletem os avanços mais recentes e as direções futuras que a tecnologia pode tomar. Portanto, é essencial estar ciente dessas tendências ao realizar qualquer pesquisa acadêmica ou científica.

\subsection{Aplicação da Indústria Química}

No contexto da indústria química, há várias empresas com soluções similares à proposta e a investigação das tendências tecnológicas emergentes é um aspecto crucial durante o processo de elaboração dessa ferramenta. 

\subsubsection{RECON desenvolvido pela ChemPlant Technology}

A ChemPlant Technology, s.r.o, fundada em 1991 é uma empresa situada na Republica Checa, especializada em fornecer soluções tecnológicas avançadas para indústrias de processos (principalmente Químicos, Petróleo & Gás e Geração e Distribuição de Energia). Uma de suas inovações mais notáveis é a ferramenta de reconciliação de dados chamada RECON. Esta ferramenta é fundamentada nos sólidos princípios físicos de balanço de massa e energia, um software interativo abrangente que oferece uma plataforma robusta para modelagem de complexas plantas industriais químicas e energéticas. Ele realiza uma variedade de cálculos, incluindo balanceamento de massa, energia e momento, bem como cálculos termodinâmicos. O principal objetivo da solução é validar dados que já foram obtidos de processos operacionais. No entanto, a ferramenta também pode ser utilizada para simular o comportamento da planta sob diferentes condições.

O software é orientado para PC e possui uma interface de usuário gráfica interativa, tornando-o fácil de usar. Os usuários definem problemas (ou tarefas) interativamente através desta interface. O RECON é capaz de equilibrar materiais e energia de componentes únicos ou múltiplos de sistemas complexos, seja em estado estacionário ou instável (dinâmico), com ou sem reações químicas (balanceamento de reatores). Além disso, ele pode realizar balanceamento de momento com base em cálculos hidráulicos de vazão em sistemas de dutos. O RECON reconcilia vazões medidas, concentrações, temperaturas e outras variáveis de processo, e calcula variáveis não medidas. Para definir um problema (ou tarefa), os usuários geralmente criam um fluxograma de processo e definem variáveis de processo, como taxas de fluxo, temperaturas, pressões, etc. O fluxograma inclui nós, fluxos de massa e energia, e trocadores de calor. Se necessário, os usuários também podem complementar (ou até mesmo substituir) o modelo de balanceamento com suas próprias equações.

\subsubsection{BILCO desenvolvida pela CASPEO}

CASPEO é uma empresa francesa fundada em 2004, especializada em engenharia de processos e soluções tecnológicas. Originada do 
Departamento de Pesquisas Geológicas e Minerais da França. Ela foi criada para oferecer à indústria de mineração métodos e ferramentas computacionais resultantes de anos de pesquisa tornou-se uma referência na indústria de processamento mineral, atendendo a vários mercados, como mineração e metalurgia, processamento de biomassa e alimentos, tratamento de resíduos sólidos e outras indústrias de processamento. 

Uma das suas inovações é o software de reconciliação de dados BILCO, projetado para derivar um balanço de material coerente e total a partir de todos os dados disponíveis (medições, análises, estimativas) para todas as correntes de processo. Ele é uma ferramenta poderosa que permite aos usuários reconciliar dados de qualquer planta de processamento.

Ele tem a capacidade de incorporar a duas outras ferramentas sem nenhum problema, um simulador de processos, denominado USIM PAC e a um software de contabilidade metalúrgica, INVETEO, e dessa forma fornece cálculos de balanço precisos, e gera um conjunto de estimadores coerentes que estão em conformidades com as restrições da lei de conservação de energia além de calcular seus erros relativos. Ele também é capaz de determinar, em quantidade e qualidade, a composição de cada corrente de processo. É um dos poucos softwares de balanço de massa capazes de calcular toda a composição da corrente (taxas de fluxo, classes de tamanho, tipos de partículas, massa molar, etc.) em um único cálculo.

Essa solução também oferece uma única interface para gerenciar todo o processo, com interface gráfica é fácil de usar, tornando-a acessível tanto para novos usuários quanto para os mais experientes. Uma das características mais úteis do BILCO é a possibilidade de exportar os resultados para o Excel, permitindo uma análise mais aprofundada dos dados. Ele fornece resultados detalhados, incluindo uma planilha de comparação e uma planilha global, para uma visão completa do balanço de material.

\subsubsection{PIMSOFT SigmaFine desenvolvida pela Shorou International}

Shorou International é uma empresa dos Emirados Árabes Unidos, que oferece soluções especializadas e serviços de engenharia com foco em automação avançada e gestão de ativos para todas as principais indústrias com ênfase especial nos setores de petróleo, gás, utilidade e energia.

Uma das suas soluções é o Sistema OSIsoft PI, uma plataforma que coleta, historiza e analisa grandes quantidades de dados de séries temporais de alta fidelidade de várias fontes de dados em diferentes formados. Esses dados são disponibilizados para usuários e sistemas em diversos setores de negócios. As implementações do sistema PI aproveitam o poder dos dados operacionais para gerar previsões que aumentam a consciência situacional e desencadeiam decisões bem planejadas, ajudando as empresas líderes a alcançar maiores melhorias operacionais e inovações revolucionárias em seus respectivos campos. 

E um complemento desse sistema, é o PIMSOFT SigmaFine, um software de verificações e balanços que utiliza princípios de conservação, estatísticas, padrões de engenharia e cálculos para monitorar e montar dados de plantas industriais. SigmaFine gera módulos coerentes, confiáveis e utilizáveis, prontos para negócios.

\subsection{Aplicação da Contabilidade}

Paralelamente à disciplina de Reconciliação de Dados na área industrial química, há uma outra área na qual investe bastante nesse campo, a de soluções contábeis. E a investigação das tendências tecnológicas emergentes é um aspecto crucial durante o processo de desenvolvimento de uma solução mais dinâmica e que consegue resolver problemas já solucionados por outras áreas da ciência.

\subsubsection{FloQast Reconciliation Management desenvolvida pela FloQast}

FloQast é uma empresa fundada em 20213 na qual é uma plataforma  de contabilidade operacional baseada na nuvem,com foco em automação e gestão para uso por contadores. Uma das suas soluções é uma ferramenta de reconciliação de dados nomeada FloQast Reconciliation Management, uma solução avançada de automação de fluxo de trabalho para fornecer gerenciamento de reconciliação de contas de ponta a ponta. Isso aumenta a velocidade e a precisão financeira do fechamento financeiro, ao mesmo tempo que gerencia o risco de declaração incorreta. Esse software permite que controladores e suas equipes automatizem e gerenciem o processo de reconciliação de ponta a ponta com uma solução centralizada confiável por contadores e auditores em todo o mundo.

\subsubsection{Aspen Unified Reconciliation and Accounting desenvolvida pela Aspen}

AspenTech, uma empresa fundada em 1981, a partir do Projeto ASPEN - uma pesquisa conjunta entre o MIT (Instituto de Tecnologia de Massachusetts) e o Departamento de Energia dos EUA na qual desenvolveu a primeira tecnologia de modelagem e simulação baseada em computador para a indústria química. Hoje, mais de 40 anos depois, com mais de 3700 funcionários e 60 localidades em todo o mundo. A empresa tem seu foco em soluções industriais, ao mesmo tempo em que enfrenta diretamente a sustentabilidade, com foco na eficiência dos recursos, transição energética, descarbonização e redução de resíduos.

Uma das suas ferramentas que utiliza soluções similares a do trabalho de conclusão de curso é a ferramenta AURA (Aspen Unified Reconciliation Accounting) uma solução que ajuda a reduzir perdas de material e aumentar margens por meio de um equilíbrio eficiente de massa e volume. Capacitando as partes interessadas a tomar melhores decisões com base em dados de produção validados e reconciliados. A arquitetura escalável do reduz o custo total de propriedade com opções rápidas de implantação na nuvem e fácil implementação e manutenção do modelo. A precisão da reconciliação é aumentada pelo \textit{Smart Solver} proprietário que resolve automaticamente os erros de densidade medidos em laboratório. O software também ajuda a atingir metas de redução de emissões ao automatizar o rastreamento, monitoramento e relatórios de emissões de gases de efeito estufa e intensidade de carbono do produto. Além disso, permite fechar o balanço mais rapidamente com fluxos de trabalho intuitivos, interpretação fácil dos resultados com visualização avançada e relatórios dinâmicos poderosos.

\subsection{Pesquisas Sendo Desenvolvidas}

Como o objetivo do DRD é ser uma ferramenta que consegue recolher avanços tecnológicos de cada área é de suma importância não só estar ciente e alinhado às tendências da própria área de aplicação como das adjacentes, mas também estar alinhado às pesquisas da área de reconciliação de dados computacional. 

\subsubsection{Controle Avançado e Supervisão de Plantas de Processamento Mineral}

A pesquisa apresentada por Daniel Hoduin, no seu livro de Controle Avançado e supervisão de plantas de processamento mineral (2020), da  ênfase nas restrições de conservação de massa e energia, que são usadas como base para o projeto de estratégias de medição, atualização do valor medido por técnicas de filtragem de erros de medição e estimativa de variáveis de processo não medidas. Como as principais variáveis numa unidade de processamento mineral são geralmente vazões e concentrações, a sua reconciliação com as leis de conservação de massa é central para as técnicas discutidas. São propostas ferramentas para três tipos diferentes de regimes operacionais: regime estacionário, estacionário e dinâmico. Esses métodos de reconciliação são baseados nos mínimos quadrados usuais e nas técnicas de filtragem de Kalman.

\subsubsection{Ferramenta Computacional para Controle de Balanços de Materiais em Redes de Distribuição de Gás Natural}

O desenvolvimento feito pelos pesquisadores Jesús David Badillo Herrera, Arlex Chaves e José Augusto Fuentes Osorio, publicado pela revista \textit{Ciencia, Tecnología y Futuro} em 2013, aplicada em um plataforma real da \textit{Corporación Centro de Desarrollo Tecnológico de GAS} oferece uma solução prática para lidar com os desafios apresentados por erros em sistemas de distribuição de gás natural. Ao combinar técnicas numéricas e estatísticas, como a Reconciliação de Dados (DR) e a Detecção de Erros Brutos (GED), a ferramenta visa aprimorar a precisão das medições, garantir conformidade com as leis de conservação de massa e identificar e corrigir erros grosseiros. A validação da ferramenta por meio de problemas da literatura e sua aplicação a uma rede de distribuição real demonstram sua eficácia em fornecer resultados reconciliados precisos. Essa ferramenta tem o potencial de aprimorar a eficiência e confiabilidade dos processos de distribuição de gás natural, minimizando perdas de receita e complicações legais.

\section{Justificativa}

A ferramenta online de reconciliação de dados baseia-se na interseção entre a indústria e tecnologia da internet 4.0. Observa-se a necessidade premente de ferramentas especializadas que possam acompanhar e potencializar essa convergência, contribuindo para a eficiência operacional, qualidade dos processos industriais e aprimoramento da competitividade das empresas. A atual revolução industrial demanda soluções tecnológicas avançadas que permitam a integração e análise eficiente de dados em tempo real. A solução DRD torna-se crucial para manter a integridade das informações, garantindo a consistência e confiabilidade necessárias para operações industriais avançadas.

A utilização de uma ferramenta online de reconciliação de dados abre caminho para a otimização de processos, unificação com outras ferramentas, redução de erros, monitoramento em tempo real e, consequentemente, aumento da eficiência operacional. Isso não apenas reduz custos operacionais, mas também posiciona as empresas em um patamar competitivo mais vantajoso. E a integração dessa ferramenta com outras permite não apenas a reconciliação de dados, mas também a extração de insights valiosos para a tomada de decisões estratégicas. Essa sinergia contribui para a inovação contínua e a adaptação rápida às mudanças no ambiente industrial.

O desenvolvimento de uma ferramenta posiciona-se em consonância com as tendências globais de transformação digital e automação industrial. Isso não apenas fortalece a competitividade das empresas no mercado nacional, mas também as prepara para atuar de maneira efetiva em uma escala internacional. Em suma, diante do cenário tecnológico atual e das demandas evolutivas da indústria, a criação de uma ferramenta online de reconciliação de dados emerge como uma resposta estratégica para potencializar a interseção entre a Internet e o setor industrial, impulsionando a eficiência e a inovação nas práticas industriais contemporâneas.

% \section{Organização do Texto}

% Este trabalho está organizado da seguinte forma: (descrever)....

% Sources

% (1) Indústria de A a Z: entenda tudo sobre a indústria brasileira - Portal .... https://www.portaldaindustria.com.br/industria-de-a-z/.
% (2) Indústria no Brasil – Wikipédia, a enciclopédia livre. https://pt.wikipedia.org/wiki/Ind%C3%BAstria_no_Brasil.
% (3) Quais são as principais indústrias no Brasil? Como operam?. https://bing.com/search?q=setor+industrial+brasileiro.
% (4) A indústria é o motor da economia brasileira | CNN Brasil. https://www.cnnbrasil.com.br/branded-content/nacional/a-industria-e-o-motor-da-economia-brasileira/.
% (5) Importância da indústria - Portal da Indústria - CNI. https://www.portaldaindustria.com.br/estatisticas/importancia-da-industria/.
% (6) O que é um Dashboard de Dados | Microsoft Power BI. https://powerbi.microsoft.com/pt-pt/data-dashboards/.
% (7) Conheça 7 modelos de dashboard perfeitos para você aplicar. https://sendpulse.com/br/blog/modelos-de-dashboard.
% (8) Exemplo: Dashboard Personalizado da Conformidade da Reconciliação. https://bing.com/search?q=Dashboard+de+Reconcilia%c3%a7%c3%a3o+de+Dados.
% (9) Preparação de Reconciliações. https://docs.oracle.com/cloud/help/pt_BR/account-reconcile-cloud/RAARC/reconcile_user_prepare_112xd90285e8.htm.
% (10) O que é reconciliação de dados? Definição, Processo, Ferramentas. https://www.guru99.com/pt/what-is-data-reconciliation.html.
% (11) Exemplo: Dashboard Personalizado da Conformidade da Reconciliação. https://docs.oracle.com/cloud/help/pt_BR/account-reconcile-cloud/ADARC/example_reconciliation_compliance_dashboard.htm.