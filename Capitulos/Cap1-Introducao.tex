\chapter{Introdução} \label{Introducao}

% balances material quantities and their individual components satisfying conservation principles.

O panorama atual sugere que o setor industrial brasileiro continua a crescer (Tomás Torezani, 2020) porém com ele deve seguir paralelamente um avanço tecnológico para que ela consiga aumentar o nível de produtividade dessas indústrias (Marcos Lisboa, 2021), dessa forma se origina o funcionalidade do trabalho de conclusão de curso DRD (Dashboard de Reconciliação de Dados), na qual ele unifica a área de análise de qualidade dos dados e a reconciliação desses mesmos com a tecnologia oriunda dos avanços tecnológicos na área de desenvolvimento web.

\section{Objetivos}

\subsection{\textit{Objetivo geral}}

Considerado a necessidade de inovação na área industrial e da união entre essas duas áreas, o objetivo deste trabalho é apresentar o desenvolvimento de um sistema online de reconciliação de dados utilizando métodos matemáticos criados por Lagrange.

\subsection{\textit{Objetivos específicos}}

Para alcançar o objetivo proposto neste trabalho, cuja finalidade é desenvolver uma ferramenta online de reconciliação de dados é necessário especificar os seguintes objetivos específicos:

a) Realizar a pesquisa de estado da arte da área de desenvolvimento de software para tal função.

b) Analisar e compreender as metodologias e algoritmos existentes na área de reconciliação de dados, destacando suas aplicações, vantagens e limitações. A pesquisa de estado da arte é crucial para identificar as tendências mais recentes, as melhores práticas e as lacunas existentes, proporcionando uma base sólida para o desenvolvimento da ferramenta online.

c) Selecionar e adaptar as tecnologias adequadas para a implementação da ferramenta online de reconciliação de dados. Isso envolverá a avaliação de linguagens de programação, frameworks e plataformas que melhor se adéquem aos requisitos específicos da aplicação, considerando aspectos como eficiência, escalabilidade e segurança.

d) Projetar a arquitetura da ferramenta, delineando os componentes principais, a interação entre eles e a lógica de funcionamento. A clareza na definição da arquitetura será essencial para garantir a eficácia operacional da ferramenta, bem como para facilitar futuras atualizações e expansões.

e) Desenvolver a ferramenta online de reconciliação de dados, implementando os algoritmos e funcionalidades identificados na pesquisa de estado da arte. Durante essa fase, é crucial garantir a usabilidade, a integridade dos dados e a eficiência do sistema, atendendo aos padrões de qualidade e performance esperados.

f) Realizar testes abrangentes para validar a eficácia e confiabilidade da ferramenta. Isso incluirá testes de integração, testes de segurança e simulações de casos práticos para verificar a capacidade da ferramenta em lidar com diferentes cenários e volumes de dados.

g) Elaborar uma documentação completa, abrangendo desde o processo de desenvolvimento até as instruções de uso da ferramenta. Uma documentação robusta e clara será essencial para facilitar a compreensão, manutenção e futuras implementações relacionadas à ferramenta de reconciliação de dados.

\section{Estado da Arte}

\subsection{Aplicação da Indústria Química}

\subsubsection{RECON desenvolvido pela ChemPlant Technology}

A ChemPlant Technology, s.r.o, fundada em 1991 é uma empresa situada na Republica Checa, especializada em fornecer soluções tecnológicas avançadas para indústrias de processos (principalmente Químicos, Petróleo & Gás e Geração e Distribuição de Energia). Uma de suas inovações mais notáveis é a ferramenta de reconciliação de dados chamada RECON. Esta ferramenta é fundamentada nos sólidos princípios físicos de balanço de massa e energia, um software interativo abrangente que oferece uma plataforma robusta para modelagem de complexas plantas industriais químicas e energéticas. Ele realiza uma variedade de cálculos, incluindo balanceamento de massa, energia e momento, bem como cálculos termodinâmicos. O principal objetivo da solução é validar dados que já foram obtidos de processos operacionais. No entanto, a ferramenta também pode ser utilizada para simular o comportamento da planta sob diferentes condições.

O software é orientado para PC e possui uma interface de usuário gráfica interativa, tornando-o fácil de usar. Os usuários definem problemas (ou tarefas) interativamente através desta interface. O RECON é capaz de equilibrar materiais e energia de componentes únicos ou múltiplos de sistemas complexos, seja em estado estacionário ou instável (dinâmico), com ou sem reações químicas (balanceamento de reatores). Além disso, ele pode realizar balanceamento de momento com base em cálculos hidráulicos de vazão em sistemas de dutos. O RECON reconcilia vazões medidas, concentrações, temperaturas e outras variáveis de processo, e calcula variáveis não medidas. Para definir um problema (ou tarefa), os usuários geralmente criam um fluxograma de processo e definem variáveis de processo, como taxas de fluxo, temperaturas, pressões, etc. O fluxograma inclui nós, fluxos de massa e energia, e trocadores de calor. Se necessário, os usuários também podem complementar (ou até mesmo substituir) o modelo de balanceamento com suas próprias equações.

\subsubsection{BILCO desenvolvida pela CASPEO}

CASPEO é uma empresa francesa fundada em 2004, especializada em engenharia de processos e soluções tecnológicas. Originada do 
Departamento de Pesquisas Geológicas e Minerais da França. Ela foi criada para oferecer à indústria de mineração métodos e ferramentas computacionais resultantes de anos de pesquisa tornou-se uma referência na indústria de processamento mineral, atendendo a vários mercados, como mineração e metalurgia, processamento de biomassa e alimentos, tratamento de resíduos sólidos e outras indústrias de processamento. 

Uma das suas inovações é o software de reconciliação de dados BILCO, projetado para derivar um balanço de material coerente e total a partir de todos os dados disponíveis (medições, análises, estimativas) para todas as correntes de processo. Ele é uma ferramenta poderosa que permite aos usuários reconciliar dados de qualquer planta de processamento.

Ele tem a capacidade de incorporar a duas outras ferramentas sem nenhum problema, um simulador de processos, denominado USIM PAC e a um software de contabilidade metalúrgica, INVETEO, e dessa forma fornece cálculos de balanço precisos, e gera um conjunto de estimadores coerentes que estão em conformidades com as restrições da lei de conservação de energia além de calcular seus erros relativos. Ele também é capaz de determinar, em quantidade e qualidade, a composição de cada corrente de processo. É um dos poucos softwares de balanço de massa capazes de calcular toda a composição da corrente (taxas de fluxo, classes de tamanho, tipos de partículas, massa molar, etc.) em um único cálculo.

Essa solução também oferece uma única interface para gerenciar todo o processo, com interface gráfica é fácil de usar, tornando-a acessível tanto para novos usuários quanto para os mais experientes. Uma das características mais úteis do BILCO é a possibilidade de exportar os resultados para o Excel, permitindo uma análise mais aprofundada dos dados. Ele fornece resultados detalhados, incluindo uma planilha de comparação e uma planilha global, para uma visão completa do balanço de material.

\subsubsection{PIMSOFT SigmaFine desenvolvida pela Shorou International}

Shorou International é uma empresa dos Emirados Árabes Unidos, que oferece soluções especializadas e serviços de engenharia com foco em automação avançada e gestão de ativos para todas as principais indústrias com ênfase especial nos setores de petróleo, gás, utilidade e energia.

Uma das suas soluções é o Sistema OSIsoft PI, uma plataforma que coleta, historiza e analisa grandes quantidades de dados de séries temporais de alta fidelidade de várias fontes de dados em diferentes formados. Esses dados são disponibilizados para usuários e sistemas em diversos setores de negócios. As implementações do sistema PI aproveitam o poder dos dados operacionais para gerar previsões que aumentam a consciência situacional e desencadeiam decisões bem planejadas, ajudando as empresas líderes a alcançar maiores melhorias operacionais e inovações revolucionárias em seus respectivos campos. 

E um complemento desse sistema, é o PIMSOFT SigmaFine, um software de verificações e balanços que utiliza princípios de conservação, estatísticas, padrões de engenharia e cálculos para monitorar e montar dados de plantas industriais. SigmaFine gera módulos coerentes, confiáveis e utilizáveis, prontos para negócios.

\subsection{Aplicação da Contabilidade}

\subsubsection{FloQast Reconciliation Management desenvolvida pela FloQast}

Shorou International é uma empresa dos Emirados Árabes Unidos, que oferece soluções especializadas e serviços de engenharia com foco em automação avançada e gestão de ativos para todas as principais indústrias com ênfase especial nos setores de petróleo, gás, utilidade e energia.

\subsubsection{Aspen Unified Reconciliation and Accounting desenvolvida pela Aspen}

Shorou International é uma empresa dos Emirados Árabes Unidos, que oferece soluções especializadas e serviços de engenharia com foco em automação avançada e gestão de ativos para todas as principais indústrias com ênfase especial nos setores de petróleo, gás, utilidade e energia.

\subsection{Pesquisas Sendo Desenvolvidas}

\subsubsection{Daniel Hoduin - Controle Avançado e Supervisão de Plantas de Processamento Mineral}

A pesquisa apresentada por Daniel Hoduin, no seu livro de Controle Avançado e supervisão de plantas de processamento mineral (2020), da  ênfase nas restrições de conservação de massa e energia, que são usadas como base para o projeto de estratégias de medição, atualização do valor medido por técnicas de filtragem de erros de medição e estimativa de variáveis de processo não medidas. Como as principais variáveis numa unidade de processamento mineral são geralmente vazões e concentrações, a sua reconciliação com as leis de conservação de massa é central para as técnicas discutidas. São propostas ferramentas para três tipos diferentes de regimes operacionais: regime estacionário, estacionário e dinâmico. Esses métodos de reconciliação são baseados nos mínimos quadrados usuais e nas técnicas de filtragem de Kalman.


\subsubsection{Computational  tool for Material balances Control in Natural Gas Distribution Network - Jesús Herrera}

A ferramenta computacional apresentada neste artigo oferece uma solução prática para lidar com os desafios apresentados por erros em sistemas de distribuição de gás natural. Ao combinar técnicas de Reconciliação de Dados e Detecção de Erros Grosseiros, a ferramenta visa aprimorar a precisão das medições, garantir conformidade com as leis de conservação de massa e identificar e corrigir erros grosseiros. A validação da ferramenta por meio de problemas da literatura e sua aplicação a uma rede de distribuição real demonstram sua eficácia em fornecer resultados reconciliados precisos. Essa ferramenta tem o potencial de aprimorar a eficiência e confiabilidade dos processos de distribuição de gás natural, minimizando perdas de receita e complicações legais.

--- Ideias 

- The utility of Bayesian data reconciliation for separations
% Mencionar outras formas de reconciliação de dados no texto acima


\section{Justificativa}

A ferramenta online de reconciliação de dados baseia-se na interseção entre a indústria e tecnologia da internet 4.0. Observa-se a necessidade premente de ferramentas especializadas que possam acompanhar e potencializar essa convergência, contribuindo para a eficiência operacional, qualidade dos processos industriais e aprimoramento da competitividade das empresas. A atual revolução industrial demanda soluções tecnológicas avançadas que permitam a integração e análise eficiente de dados em tempo real. A solução DRD torna-se crucial para manter a integridade das informações, garantindo a consistência e confiabilidade necessárias para operações industriais avançadas.

A utilização de uma ferramenta online de reconciliação de dados abre caminho para a otimização de processos, unificação com outras ferramentas, redução de erros, monitoramento em tempo real e, consequentemente, aumento da eficiência operacional. Isso não apenas reduz custos operacionais, mas também posiciona as empresas em um patamar competitivo mais vantajoso. E a integração dessa ferramenta com outras permite não apenas a reconciliação de dados, mas também a extração de insights valiosos para a tomada de decisões estratégicas. Essa sinergia contribui para a inovação contínua e a adaptação rápida às mudanças no ambiente industrial.

O desenvolvimento de uma ferramenta posiciona-se em consonância com as tendências globais de transformação digital e automação industrial. Isso não apenas fortalece a competitividade das empresas no mercado nacional, mas também as prepara para atuar de maneira efetiva em uma escala internacional. Em suma, diante do cenário tecnológico atual e das demandas evolutivas da indústria, a criação de uma ferramenta online de reconciliação de dados emerge como uma resposta estratégica para potencializar a interseção entre a Internet e o setor industrial, impulsionando a eficiência e a inovação nas práticas industriais contemporâneas.

\section{Organização do Texto}

Este trabalho está organizado da seguinte forma: (descrever)....

%Sugestões para estrutura da monografia:

% \begin{enumerate}[label=(\alph*)]
%   \item Introdução
%   \item Referencial Teórico (ou Revisão Bibliográfica)
%   \item Materiais e Métodos (ou Metodologia)
%   \item Resultados e Discussões
%   \item Conclusão (ou Considerações Finais)
%   \item Referências
% \end{enumerate}