\chapter{Metodologia} \label{metodologia}

\section{Reconciliação de Dados no Âmbito Industrial}
% Reconciliação de Dados
% ========================================================
% QUESTÃO:

Já no começo da década de 60 se entendeu a importância do controle de processos químicos industriais por computadores utilizando técnicas matemáticas (KUEHN, 1961), dessa forma surge a área da computação voltada à reconciliação de dados, na qual há a comparação, validação e correção de informações coletadas em diferentes pontos do processo, afim de determinar a consistência dos dados, a qualidade dos mesmos ou até otimizar os processos (NARASIMHAN, 2000).

\clearpage
